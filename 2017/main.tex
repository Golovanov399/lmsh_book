\documentclass[a5paper,10pt]{article}
\usepackage[utf8]{inputenc}
\usepackage[T2A]{fontenc}
\usepackage[russian]{babel}
\usepackage{amsmath,amssymb}
\usepackage{jeolm}
\usepackage{jeolm-tourn-limited}
\usepackage{parskip}
\usepackage{epigraph}
\pagestyle{empty}

\usepackage[margin=2em]{geometry}
\usepackage{hyperref}
\hypersetup{
	colorlinks,
	citecolor=black,
	filecolor=black,
	linkcolor=black,
	urlcolor=black
}
\usepackage{pgfpages}

%\pgfpagesuselayout{2 on 1}[a4paper,landscape]

% Перенос знаков в формулах (по Львовскому)
\newcommand*{\hm}[1]{#1\nobreak\discretionary{}
{\hbox{$\mathsurround=0pt #1$}}{}}

\renewcommand{\le}{\leqslant}
\renewcommand{\ge}{\geqslant}

\def\jeolmevent{Кировская летняя многопредметная школа}
\def\jeolmcity{Вишкиль}
\def\jeolmdaterange{3--28 июля 2017 года.}

\begin{document}

\jeolmheader
\begin{center}
\textbf{\Large Параболы}\\
\textit{8 класс}\\
\textit{05.07.17}
\end{center}

\begin{problems}
\item На рисунке изображены графики трёх квадратных трёчленов. Можно ли подобрать такие числа $a, b$ и $c$, чтобы это были графики трёхчленов $ax^2 + bx + c$, $bx^2 + cx + a$  и $cx^2 + ax + b$?
%http://www.problems.ru/view_problem_details_new.php?id=109457

\begin{center}
\includegraphics[scale=0.8]{simple/parabola.jpg}
\end{center}

\item Квадратный трехчлен $y = ax^2 + bx + c$  не имеет корней и $a + b + c > 0$.  Найдите знак коэффициента $c$.
%геометрическая интерпретация

\item Верно ли, что если $b > a + c > 0$, то квадратное уравнение $ax^2 + bx + c = 0$ имеет два корня?
%геометрическая интерпретация

\item Дан многочлен $P(x) = t^2 - 4t$. Доказать, что при любых $x \ge 1$ и $y \ge 1$ выполняется $P(x^2+y^2) \ge P(2xy)$. 

\item Существуют ли такие три квадратных трёхчлена, что каждый из них имеет хотя бы один корень, а сумма любых двух из них корней не имеет?

\item 

\item Приведенные квадратные трёхчлены $f(x)$ и $g(x)$ таковы, что уравнения  $f(g(x))\hm= 0$  и  $g(f(x)) = 0$  не имеют вещественных корней. Докажите, что хотя бы одно из уравнений  $f(f(x)) = 0$  и  $g(g(x)) = 0$  тоже не имеет вещественных корней.
\end{problems}
\resetproblem
\newpage

%\jeolmheader
%\input{simple/2017_07_06_neq_numbers.tex}
%\input{simple/2017_07_05_geom_schet_otrezkov.tex}
%\resetproblem
%\newpage

%\jeolmheader
%\input{simple/2017_07_06_neq_numbers.tex}
%\input{simple/2017_07_07_geom_orto_i_9tochek.tex}
%\resetproblem
%\newpage
\end{document}


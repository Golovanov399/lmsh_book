\begin{center}
\textbf{\Large Программа теоретического зачета}\\
% \textit{Обычные группы, 7 класс}\\
\textbf{\textit{Алгебра и теория чисел}}

\end{center}

\begin{problems}
\item Линейное представление НОД. Доказательство линейного представления через катящееся колесо.

\item Алгоритм Евклида.

\item Линейное представление НОД. Доказательство линейного представления через алгоритм Евклида.

\item Сформулируйте и докажите свойства сравнений.

\item Докажите, что если $a  \mathop{\equiv}\limits_m b,$   то $a^k  \mathop{\equiv}\limits_m b^k,$  где $k$~--- натуральное число. Cформулируйте, когда можно сокращать на одно и то же число обе части сравнения, и докажите это свойство.

\item Пусть $a$~--- некоторое число, которое не делится на простое число $p$. Докажите, что в последовательности $0\cdot a$, $1\cdot a$, $2\cdot a,\dots, (p-1)\cdot a$ все числа дают разные остатки по модулю $p$. Сформулируйте и докажите, используя этот факт, малую теорему Ферма.

\item Пусть $p$~--- простое число. Докажите, что для любых чисел $a$ и $b$ верно, что $(a+b)^p \mathop{\equiv}\limits_p a^p+b^p$. Выведите из этой задачи малую теорему Ферма.

\item Отметим на бумаге произвольным образом $p-1$ точку. Каждой точке сопоставим какой-то ненулевой остаток при делении на $p$. Проведём из остатка $k$ стрелочку в остаток $ka$. Докажите, что все точки разбиваются на  циклические маршруты одинаковой длины. Выведите малую теорему Ферма.

\item Определение НОК и НОД. Их свойства. 

\item Обратные остатки.

\item Теорема Вильсона. Её доказательство. Теорема, обратная теореме Вильсона.
\end{problems}

\resetproblem

\begin{center}
\textbf{\textit{Классическая комбинаторика}}
\end{center}
\begin{problems}
\item Докажите комбинаторно и алгебраически: $kC_n^k=nC_{n-1}^{k-1}.$

\item Докажите комбинаторно и алгебраически: $C_n^{k-1}+C_n^k=C_{n+1}^k.$

\item Два определения треугольника Паскаля и их равносильность.

\item Бином Ньютона и его связь с треугольников Паскаля.

\item Любые два доказательства тождества: $C_n^0+C_n^1+\cdots+C_n^n=2^n.$

\item Любые два доказательства тождества: $C_n^0-C_n^1+C_n^2-\cdots\pm C_n^n=0.$

\item Докажите, используя треугольник Паскаля: $C_k^k+C_{k+1}^k+\cdots+C_n^k=C_{n+1}^{k+1}.$

\item Докажите, используя треугольник Паскаля: $(C_n^0)^2 + (C_n^{1})^2+\cdots+(C_n^n)^2=C_{2n}^n.$

\item Пусть $p>2$~--- простое число. Сколько существует способов раскрасить вершины правильного $p$-угольника в $a$ цветов (раскраски, которые можно совместить поворотом, считаются одинаковыми)? Выведите из этого малую теорему Ферма. 

\item Пусть $p>2$~--- простое число. Сколькими способами можно провести через вершины правильного $p$-угольника замкнутую ориентированную $p$-звенную ломаную (ломаные, которые можно совместить поворотом, считаются одинаковыми)? Выведите из этого теорему Вильсона.
\end{problems}

\resetproblem

\begin{center}
\textbf{\textit{Геометрия}}
\end{center}
\begin{problems}
\item Внутри треугольника взяли две произвольные точки. Докажите, что расстояние между ними не превосходит наибольшей стороны треугольника.

\item \textbf{Задача Герона из Александрии.} Точки $A$ и $B$ лежат по одну сторону от прямой. Постройте на этой прямой такую точку $M$, чтобы сумма отрезков $AM+BM$ была минимальна. 

\item \textbf{Рельсы Евклида.} Формулировка и доказательство.

\item \textbf{Лемма о линолеуме.} Докажите, что если пол в комнате площадью $S$ надо покрыть линолеумом общей площадью также $S$ так, чтобы не было участков, покрытых более чем в два слоя, то площадь пола, покрытая дважды, равна площади пола, не покрытой ни разу. 

\item ГМТ, равноудаленных от данных точек $A,B$ и $C$. ГМТ, равноудаленных от данных прямых $a, b$ и $c$.

\item Четвёртый признак равенства треугольников. Доказательство того, что в треугольниках $ABC$ и $A'B'C'$ с $AB=A'B'$ и $\angle BAC+\angle B'A'C'=180^{\circ}$ равенство сторон $BC$ и $B'C'$ равносильно равенству углов $\angle BCA$ и $\angle B'C'A'$.

\item Внутри острого угла с вершиной $O$ взяли произвольную точку $A$. Её отразили относительно сторон угла и получили точки $A_1$ и $A_2$. Докажите, что угол $A_1OA_2$ не зависит от выбора точки.

\item Высоты треугольника пересекаются в одной точке. Ортоцентрическая четвёрка.
\end{problems}

\resetproblem

\begin{center}
\textbf{\textit{Теория графов}}
\end{center}
\begin{problems}
\item Определение двудольного графа.\\ а) Доказательство того, что все циклы в двудольном графе имеют чётную длину.\\ 
б) Доказательство того, что если в двудольном графе есть замкнутый цикл, проходящий через каждую вершину ровно по~одному разу, то~вершин каждого цвета поровну.\\ 
в) Доказательство того, что если в двудольном графе есть путь, проходящий через каждую вершину ровно по~одному разу, то~число белых вершин отличается от~числа чёрных вершин не~более, чем на~1.

\item Четыре определения дерева. Эквивалентность трёх из них.

\item Доказательство того, что в каждом дереве из более чем одной вершины есть хотя бы два листа.

\item Четыре определения дерева. Доказательство, что связный граф является деревом тогда и только тогда, когда у него $n$ вершин и $n-1$ ребро.

\item \textbf{Лемма о существовании остовного дерева (скелета).} Докажите, что из каждого связного графа можно удалить некоторое число ребер так, чтобы получилось дерево. 

\item Доказательство того, что в связном графе можно выбрать одну из вершин так, чтобы после ее
удаления вместе со всеми ведущими из неё ребрами останется связный граф.

\item Докажите, что в турнире (полном ориентированном графе) найдется путь, который проходит по каждой вершине ровно один раз. 

\item Докажите, что граф двудольный, если в нем нет циклов нечетной длины, используя индукцию по количеству вершин.

\item Докажите, что граф двудольный, если в нем нет циклов нечетной длины, используя индукцию по количеству ребер.

\item В графе степень каждой вершины не превосходит $d$. Докажите, что все вершины графа можно покрасить в $d + 1$ цвет так, чтобы любые две вершины, соединенные ребром, имели разный цвет. 

\end{problems}

\end{document}
\begin{center}
\textbf{\Large Индукция}\\
%\textit{Профи}\\
\textit{09.07.16}
\end{center}

\epigraph{\includegraphics[width=.8\textwidth]{ind.jpg}}{http://xkcd.com/1516/}

\begin{problems}
\item Из квадрата $2^n \times 2^n$ вырезали одну клетку. Докажите,
что полученную фигуру можно разрезать на уголки из трёх клеток.

\item
На сколько частей делят плоскость $n$ прямых, среди которых нет
параллельных и никакие три не пересекаются в одной точке?

\item
Докажите, что $1 + 2 + 3 + \ldots + n = \dfrac{n(n+1)}{2}$.

\end{problems}

\resetproblem

%\q4.
%ханойские башни

%\q5.
%Докажите, что $3^{2n+2}+8n-9$ делится на 16.

%\q6.
%{\itshape Утверждение.} У всех людей глаза одинакового цвета. \\
%{\itshape Доказательство. } Докажем, что в компании из $n$  людей у всех глаза одинакового цвета, методом
%математической индукции.
%При     $n=1$    утверждение очевидно. \\

%Предположим, что в любой компании из $k$  человек у всех глаза одинакового цвета. Рассмотрим
%компанию из   $k+1$   человека. Перенумеруем их: первый, второй, третий, …, $k$ -ый,    $(k+1)$ -ый. По
%предположению, у любых   из этих людей глаза одинакового цвета. Значит, у первого, второго,
%…,  $k$-го — глаза одинакового цвета. \\

%Возьмём теперь другую группу из человек: второй, третий, …,  $k$-ый, $(k+1)$-ый. У них тоже
%глаза одинакового цвета. В обеих группах присутствовал второй человек. Значит у всех них глаза
%того же цвета, как и у второго, то есть у всех одинаковые. {\bf Где же ошибка?}


{\bf Задачи для самостоятельного решения.}


\begin{problems}
%равенства
\item
Докажите, что $1^2 + 2^2 + \ldots + n^2 = \dfrac{n(n+1)(2n+1)}{6}.$

%\item
%Докажите, что $1^3+2^3+3^3+ \ldots + n^3 = %\dfrac{n^2(n+1)^2}{4}$

\item
Докажите, что $1 \cdot 1! + 2 \cdot 2! + \ldots + n \cdot n! = (n+1)! - 1.$

%\item
%Докажите, что для любого $x \ne 1$ $1 + x + x^2 + %\ldots + x^n =
%\frac{x^{n+1} - 1}{x - 1}$

\item
Докажите, что $\frac{1}{1 \cdot 2} + \frac{1}{2 \cdot 3} + \ldots + \frac{1}{n \cdot (n+1)} = 1 - \frac{1}{n+1}.$

%\item
%Докажите равенство $3 + 33 + 333 + \ldots + \underbrace{33 \ldots
%33}_n = \dfrac{10^{n+1} - 9n - 10}{27}$

%\item
%Докажите, что
%$$ (1 - \frac{1}{4}) (1 - \frac{1}{9}) \ldots (1 - \frac{1}{n^2}) = \frac{n+1}{2n}$$

%полегче

\item
У бородатого многоугольника во внешнюю сторону растет щетина. Его
пересекает несколько прямых, на каждой из которых с одной из сторон
тоже растет щетина. В результате многоугольник оказался разбитым на
некоторое число частей. Докажите, что хотя бы одна из частей
окажется бородатой снаружи (никакие три прямые не проходят через
одну точку).

\item
Плоскость поделена на области несколькими прямыми. Докажите, что эти
области можно раскрасить в два цвета так, чтобы любые две соседние
области были раскрашены в различные цвета.



%средние



\item
Докажите, что если клетчатая доска $n \times n$ покрашена в 4 цвета
так, что любой квадрат $2 \times 2$ содержит все четыре цвета, то:
если $n$~--- чётно, то в углах квадрата стоят разные цвета; если $n$~---
нечётно, то в углах квадрата стоят одинаковые цвета.

%\item
%Кусок бумаги разрешается рвать на 4 или на 6 кусков. Докажите, что по этим правилам его можно разорвать на любое число кусков, начиная с шести.

%\item Докажите, что любой квадрат можно разрезать на $n$ квадратов, начиная с шести.


\item
Пусть $x$~--- такое, что $x+ \frac{1}{x}$~--- целое. Докажите, что
$x^n + \frac{1}{x^n}$~--- целое для любого натурального $n$.

%сложные



\item
Доказать, что для любого натурального $n$ существует $n$-значное число, составленное из цифр $1$ и $2$, которое делится на $2^n$.

\item
В выпуклом многоугольнике некоторые стороны и диагонали окрашены в
красный цвет так, что никакие две красные диагонали не пересекаются
внутри многоугольника, а любые две вершины многоугольника соединены
ломаной из красных отрезков. Докажите что сумма длин всех красных
отрезков больше полупериметра многоугольника.

\item 
Есть два квадрата со стороной $25\cdot 2^n$. В одном Василий Иванович отмечает одну клетку. Петька должен разрезать свой квадрат на части так, что:

(a) среди них есть квадратик со стороной $1$;

(b) какую бы клетку Василий Иванович не отметил бы, Петька сможет закрыть своими фигурками все клетки квадрата Чапаева кроме отмеченной. 

Докажите, что Петьке достаточно разбить квадрат на $6+n$ деталей.



%\item
%Пусть $p$ -- простое и $gcd(n,p) = 1$. Докажите, что $n^{p-1}$ даёт остаток 1 при делении на $p$.

%\item На сколько частей делят пространство $n$ плоскостей,если никакие три из которых не пересекаются по одной прямой и никакие 4 из них не имеют общей точки? (на какое наибольшее число частей могут разбивать плоскость $n$ окружностей)

%\item есть жесть

%делимость

%\item Докажите, что $5 \cdot 2^{3n+1} + 3^{3n+2}$ делится на 19.

%\item Числа вида $F_n = 2^{2^n} + 1$ называются числами Ферма. Докажите, что десятичная запись числа $F_n$ при $n \geqslant 2$ оканчивается цифрой 7.

%\item Докажите, что все числа 10017, 100117, 1001117, . . . делятся на 53.

%\item Докажите, что $2^{3^n}$ + 1 делится на $3^{n+1}$.

%\item Докажите, что $3^{2^n} - 1$ делится на $2^{n+2}$, но не делится на $2^{n+3}$.

\end{problems}
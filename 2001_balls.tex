\begin{center}
\textbf{\Large И снова комбинаторика}\\
%\textit{Профи}\\
\textit{20.07.16}
\end{center}

\renewcommand{\epigraphwidth}{.45\textheight}

\epigraph{\it Двумерная сфера~--- это граница трёхмерного шара, но гомотопически трёхмерный шар примитивен, он стягивается в точку, а сфера являет собой все тайны гомотопического хаоса.}{Роман Михайлов}

\begin{problems}

\item 
(а) Есть буквы Ш (12 штук) и П (5 штук). Сколькими способами можно их расставить?

(b) Есть Шары (12 штук) и Перегородки (5 штук). Сколькими способами можно их переставить?
 
(c) Есть 6 коробок и 12 одинаковых кубиков. Сколькими способами можно разложить кубики по коробкам?
 
(d) Игральную кость подкидывают 12 раз. Сколько различных вариантов есть?

\item 
(а) В магазине 5 касс и всего 10 покупателей. Сколькими способами они могут распределиться по очередям (очередь у любой кассы может быть любой длины, все кассы разные)? 

(b) В почтовом отделении продаётся 5 видов открыток. Сколькими способами можно купить 10 открыток?
 
(c) Сколько решений в натуральных числах имеет уравнение: $x+y+z+a+b=10$.

%\item (a) Тайным голосованием 30 человек голосуют по 5 предложениям. Сколькими способами могут распределиться голоса, если каждый голосует только за одно предложение и учитывается лишь количество голосов, поданных за каждое предложение?

%(b) Та же задача, но голосование открытое --- важно, кто за кого голосует.

% вообще, можно бы было поговорить сначала про деление на 2.
\item
(a) Есть 10 ромашек, 15 васильков и 14 незабудок.
Сколькими способами можно раздать цветы трём различным черепашкам?

(b) Оказалось, что букеты предназначены для завтрака этих черепашек. Каждой черепашке важно, в каком порядке она ест цветы. Сколькими способами можно покормить черепашек?

(c) Сколько имеет решений в натуральных числах уравнение $xyz=2^{10}3^{14}7^{15}$

\item 
(a) Сколькими способами можно разместить $n$ различных флагов на $k$ различных мачтах, если все флаги должны быть развешаны, а конфигурации, отличающиеся порядком флагов на матче, считаются одинаковыми?

(b) Тот же вопрос, но конфигурации, отличающиеся порядком флагов на мачте, считаются разными.

(с) Есть $n$ различных букв и $k-1$ одинаковая буква. Сколько слов можно составить из этого набора, если обязательно использовать все буквы.

\item (a) Сколькими способами 3 человека могут разделить между собой 2016 одинаковых яблок, 1 апельсин, 1 сливу, 1 мандарин, 1 грушу, 1 айву и 1 хурму?

(b) Черепашка Пося съела 24 яблок. Сколькими способами можно поровну разделить оставшиеся фрукты между тремя людьми?

% я понимаю, что эта задача не совсем про то, но блин.
%\item Заданы числа $M$ и $G$, причём $M=GA^aB^bC^cD^d$ и $A$, $B$, $C$, $D$ ---простые числа. Сколькими способами можно выбрать $X$ и $Y$, что $(X, Y)=G$, $[X, Y]=M$.

%\item В шахматной олимпиаде участвует по 4 представителя от $n$ стран. Сколькими способами они могут стать в ряд так, что рядом с каждым был бы представитель этой же страны?

\end{problems}

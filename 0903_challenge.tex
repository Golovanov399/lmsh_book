\begin{problems}

\item Сколько есть 6-значных чисел, в записи которых есть хотя бы одна чётная цифра?
%900000 - 5^6
%дополнение

\item Сколькими способами колоду из 36 карт можно перетасовать так, чтобы красные и чёрные карты
чередовались?
%2\cdot(18!)^2
%произведение 

%\item Сколько существует слов длины $n$ в алфавите из $m$ букв таких, что они содержат слово $A$ в качестве подпоследовательности и начинаются на ту же букву, что и $A$?
%C_{n-1}^{|A|-1}\cdot(m - 1)^{n - |A|}

%\item Сколько существует 6-значных чисел, в которых чётных и нечётных цифр поровну?
%C_5^2\cdot(%5^6 + 4\cdot5^5)
%цешка
%\item Сколько имеется 7-значных чисел, у которых каждая следующая (слева направо) цифра не больше предыдущей?
%C_{16}^7 - 1
%цешка

\item Сколькими способами можно выстроить 8 человек в очередь так, чтобы Иванов, Петров и Сидоров
стояли подряд?
%6!\cdot3!
%цешка

\item Преподаватель знает 7 задач по комбинаторике, 8 задач на геометрию и 9 задач по алгебре. Сколькими способами он может составить домашнее задание для студентов из 6 задач, не все из которых на одну тему? Порядок задач не имеет значения.
%C_{24}^6 - C_7^6 - C_8^6 - C_9^6
%дополнение

\item Сколькими способами можно образовать 6 пар из 12 человек?
%12! / 2^6
%произведение

\item Сколькими способами можно выписать в ряд цифры от 0 до 9 так, чтобы четные цифры шли в порядке возрастания, а нечетные — в порядке убывания?
%C_{10}^5
%цешка 

\item На плоскости отмечено 10 точек так, что никакие три из них не лежат на одной прямой. Сколько существует треугольников с вершинами в этих точках?
%C_{10}^3
%тоже цешка

\item Назовем натуральное число ``симпатичным'', если в его записи встречаются только нечетные цифры. Сколько существует четырехзначных ``симпатичных'' чисел? 
%5^4
%произведение

\item Сколько существует шестизначных чисел, все цифры которых имеют одинаковую четность?
%5^6 + 4\cdot5^5
%произведение

\item Дан шестизначный номер телефона. Из скольких семизначных номеров его можно получить вычеркиванием одной цифры? 
%63
% произведение

\item Сколько четырехзначных чисел можно составить, используя цифры 1, 2, 3, 4 и 5, если:

  а) никакая цифра не повторяется более одного раза;
  
  б) повторения цифр допустимы;
  
  в) числа должны быть нечетными и повторений цифр быть не должно?
%5!, 5^4, 3\cdot4\cdot3\cdot2
% произведение

\item Из двух математиков и десяти экономистов надо составить комиссию из восьми человек. Сколькими способами можно составить комиссию, если в нее должен входить хотя бы один математик? 
%C_{12}^8 - C_{10}^8
%дополнение

%\item План города имеет схему, представляющую собой прямоугольник $5 \times 10$ клеток. На улицах введено одностороннее движение: разрешается ехать только вправо и вверх. Сколько есть различных маршрутов, ведущих из левого нижнего угла в правый верхний?
%C_{15}^5

%\item Улитка должна проползти вдоль линий клетчатой бумаги путь длины $2n$, начав и кончив свой путь в данном узле. Найдите число различных маршрутов улитки.

%\item Улитка умеет ползать вдоль линий клетчатой бумаги только вверх и вправо. Найдите количество различных путей из точки с координатой $(0,0)$ в точку с координатой $(n,m)$. 
%C_{n+m}^m

%\item Каких чисел среди всех целых чисел от 1 до 1000000 больше и на сколько: делящихся на 5, чья сумма цифр на 5 не делится, или не делящихся на 5, чья сумма цифр на 5 делится?

\item Сколько существует девятизначных чисел, сумма цифр которых чётна?
%9\cdot10^8/2
%произведение

\item Сколькими способами можно выбрать на шахматной доске черный и белый квадраты, не лежащие на одной и той же горизонтали или вертикали?
%32\cdot25, 64*49
%произведение

%\item Сколько существует чисел от $100$ до $10000$, в записи которых встречаются ровно три одинаковые цифры?

%\item Из колоды, содержащей $52$ карты, вынули $10$ карт. Во скольких случаях среди этих карт есть ровно один туз?

%\item Сколькими способами можно разложить $12$ пятаков по $5$ различным кошелькам так, чтобы ни один кошелек не оказался пустым?

%\item Сколько можно составить ожерелий из пяти одинаковых бусинок и двух большего размера?

%\item Сколькими способами можно расставить белые фигуры (два коня, два слона, две ладьи, ферзя и короля) на первой линии шахматной доски?
%произведение цэшек

\item В урне $m$ белых и $n$ черных шаров. Сколькими способами можно выбрать из урны $r$ шаров, из которых белых будет $k$ штук?
%C_m^k*C_n^{r-k}

%\item Хромая ладья ходит по клетчатой доске на одну клетку вправо или на одну клетку вверх. Найдите количество путей, ведущих из клетки \(\left(1 , 1\right)\) в клетку \(\left(m, n\right)\).

\item В колоде $36$ карт, из них четыре туза. Сколькими
способами можно сделать выбор шести карт так, чтобы среди них
было ровно два туза?
% С_4^2*C_32^2
% цешки
\end{problems}

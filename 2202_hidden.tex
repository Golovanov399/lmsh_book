\begin{center}
\textbf{\Large Разнобой}\\
%\textit{Профи}\\
\textit{22.07.16}
\end{center}

\epigraph{\it Дорогие друзья! Пришло время прощаться. Спасибо всем, кто пришёл сюда в этот зимний снежный день! Мы вас любим.}{С. Калугин}

\begin{problems}
\item На некоторых белых клетках шахматной доски стоят короли. Докажите, что их можно покрасить в два цвета так, чтобы короли одинакового цвета друг друга не били.

\item Выписаны 1000 целых чисел. Докажите, что их можно покрасить в два цвета так, чтобы отношение чисел одинакового цвета не было простым числом. 

\item В классе 30 учеников, у каждого ровно по 2 друга. Докажите, что можно организовать не менее 10 дежурств так, чтобы дежурили по двое друзей, и никто не дежурил дважды. Всегда ли можно организовать 11 дежурств?

\item Шах разбил свой квадратный одноэтажный дворец на 64 одинаковые квадратные комнаты, разделил комнаты на семь квартир (проделав двери в некоторых перегородках между комнатами) и в каждой квартире поселил по жене. Жены могут ходить по всем комнатам своей квартиры, не заходя к другим. Какое наименьшее число дверей пришлось проделать во внутренних стенах?

\item Назовем крокодилом шахматную фигуру, ход которой заключается в прыжке на $m$ клеток по вертикали или по горизонтали и затем на $n$ клеток в перпендикулярном направлении. Докажите, что для любых $m$ и $n$ можно так раскрасить клетчатую доску $1000000\times 1000000$ в два цвета (для каждых конкретных $m$ и $n$ своя раскраска), что любые две клетки, соединенные одним ходом крокодила, будут покрашены в разные цвета.

%\item Петя поставил на доску $50 \times 50$ несколько фишек, в каждую клетку~--- не больше одной. Докажите, что у Васи есть способ поставить на свободные поля этой же доски не более 99 новых фишек (возможно, ни одной) так, чтобы по-прежнему в каждой клетке стояло не больше одной фишки, и в каждой строке и каждом столбце этой доски оказалось чётное количество фишек.
%В норке живёт семья из 24 мышей. Каждую ночь ровно четыре из них отправляются на склад за сыром. Может ли так получиться, что в некоторый момент времени каждая мышка побывала на складе с каждой ровно по одному разу?
%Дана ладья, которой разрешается делать ходы только длиной в одну клетку. Доказать, что она может обойти все клетки прямоугольной шахматной доски, побывав на каждой клетке ровно один раз, и вернуться в начальную клетку тогда и только тогда, когда число клеток на доске чётно.

%Дано натуральное число  $n \geq 2$.  Рассмотрим все такие покраски клеток доски n×n в $k$ цветов, что каждая клетка покрашена ровно в один цвет и все $k$ цветов встречаются. При каком наименьшем $k$ в любой такой покраске найдутся четыре окрашенных в четыре разных цвета клетки, расположенные в пересечении двух строк и двух столбцов?
%\item Каждая деталь конструктора <<Юный паяльщик>>~--- это скобка в виде буквы П, состоящая из трёх единичных отрезков. Можно ли из деталей этого конструктора спаять полный проволочный каркас куба $2\times2\times2$, разбитого на кубики $1\times1\times1$? (Каркас состоит из 27 точек, соединенных единичными отрезками; любые две соседние точки должны быть соединены ровно одним проволочным отрезком.)
\end{problems}

\tolerance = 1600

\begin{center}
\textbf{\Large МеждусоБой}\\
\textit{12.07.16}
\end{center}

\begin{enumerate}
\item Через вершину $A$ параллелограмма $ABCD$ провели прямую $\ell$, пересекающую сторону $BC$. Докажите, что расстояние от точки $D$ до прямой $\ell$ равно сумме расстояний от точек $B$ и $C$ до этой же прямой. 
%легкая

\item В $10$ коробках лежат камни: в $k$-ой коробке лежит $2016+k$ камней ($k=~1,2,\ldots,10$). Двое играют в такую игру: один из игроков выбирает любые $5$ коробок и вынимает из них некоторое ненулевое количество камней (возможно, для разных коробок число вынутых камней различается, но для каждой из этих пяти коробок оно ненулевое). Тот, кто не может сделать ход, проигрывает. Кто из игроков выигрывает при правильной игре?
%средне-легкая 

\item На сторонах $AB$ и $BC$ равностороннего треугольника $ABC$ взяты точки $D$ и $E$ такие, что $AD = BE$. Отрезки $CD$ и $AE$  пересекаются в точке $O$. Серединный перпендикуляр к отрезку  $CO$ пересекает прямую $AO$ в точке $K$. Докажите, что $BK$ и $CO$ параллельны.
%средняя

\item По окружности расставлены в некотором порядке числа от 1 до 100. 
Назовем пару чисел {\it хорошей}, если эти два числа не стоят рядом, и хотя бы 
на одной из двух дуг, на которые они разбивают окружность, все числа 
меньше каждого из них. Чему может равняться общее количество хороших пар?
%тяжелая

\item На доске написано число 2000. Разрешается переставить в нем произвольным образом первые три цифры (ставить цифру 0 на первое место нельзя) или прибавить 361. Через 100 таких операций вновь получили 2000. Сколько раз прибавляли 361, если известно, что это сделали хотя бы один раз? 
%средне-тяжелая

\item Клетки шахматной доски $8 \times 8$ раскрашены в белый и черный цвета таким образом, что в каждом квадрате $2 \times 2$ половина клеток черные и половина белые. Сколько существует таких раскрасок?
%средне-тяжелая

\item В ряд выписано 25 цифр. Разрешается ставить между ними плюс, минус, умножить и скобки (склеивать цифры нельзя). Докажите, что всегда можно получить ноль.
%тяжелая

\item Компания из нескольких друзей вела переписку так, что каждое письмо получали все, кроме отправителя. Каждый написал одно и то же количество писем, в результате чего всеми вместе было получено 440 писем. Сколько человек могло быть в этой компании?
%легкая
\end{enumerate}

\begin{center}
\Large

\textbf{НОД. Алгоритм Евклида. 09 Июля.}
\end{center}  

\large
\def\q#1.{{\smallskip\bf #1.}}

\bigskip
\large

\begin{definition}
\emph{Алгоритм Евклида.}

Пусть a и b — целые числа, не равные одновременно нулю, и последовательность чисел
$a > b > r_1 > r_2 > r_3 > r_4 > \cdots > r_n$
определена тем, что каждое $r_k$~-- это остаток от деления предпредыдущего числа на предыдущее, а предпоследнее делится на последнее нацело, то есть \\
$a = bq_0 + r_1 \\
b = r_1q_1 + r_2 \\
r_1 = r_2q_2 + r_3 \\
\cdots \\
r_{k-2} = r_{k-1} q_{k-1} + r_k \\
\cdots \\
r_{n-2} = r_{n-1}q_{n-1}+ r_n \\
r_{n-1} = r_n q_n$ \\
Тогда $(a,b)=r_n$.
\end{definition}



\q0т. а) Докажите, что для любых целых $a, b, k$ выполнено $(a, b) = (b, a-kb)$.\\
 б) Докажите, что алгоритм Евклида действительно выдаёт наибольший общий делитель двух чисел.\\





\q1. При каких $n$ дробь $\frac{n^2+2n+4}{n^2+n+3}$ несократима?

\q2. Докажите, что $(a^{n}-1,a^{m}-1)=a^{(n,m)}-1$.

\q3. Какое наибольшее значение может принимать $(15a+16b, 16a-15b)$?

\q4. a) Найдите $(f_{n},f_{m})$, где $f_{k}=2^{2^{k}}+1$ - числа Ферма. \\ 
b) Докажите, что число $2^{2^{k}}-1$ имеет хотя бы $k$ различных простых делителей. \\
с) Из этого докажите, что простых чисел бесконечно много.

\q5т. {\bf Теорема о линейном представлении НОД.} Докажите, что существуют такие целые $x$ и $y$ , что $xa+yb=(a,b)$

\q6. Пусть $ab$ делится на $c$, $(a,c)=1$. Докажите, что $b$ делится на $c$. (Основной теоремой арифметики, разумеется, пользоваться нельзя.)

\q7т. {\bf Основная теорема арифметики.} Докажите, что каждое натуральное число $n>1$ можно единственным способом представить в виде $n=p_{1}^{d_{1}}\cdot \dots \cdot p_{k}^{d_{k}}$, где $ p_{1}<\dots <p_{k}$ — простые числа, а $d_{1}\cdot \dots \cdot d_{k}$ - натуральные числа.

\q8. По бесконечной шахматной доске ходит $(m,n)$-мамонт, который за один ход может сдвинуться на $m$ клеток по горизонтали или вертикали, а затем - на $n$ клеток в перпендикулярном направлении. При каких $m, n$ $(m,n)$-мамонт сможет из любой клетки доски попасть в любую другую?

\q9. Аня нашла себе интересное занятие. Она написала на доске две единички, потом между ними написала их сумму. Ее это так захватило, что она продолжила: брала ряд чисел, который у нее получился на предыдущем шаге, и между двумя соседними числами писала их сумму (старые числа при этом не стирала). Докажите, что любое $n>1$ она написала ровно $\varphi(n)$ раз.  Напомним, что $\varphi(n)$ -- это количество чисел, не превосходящих $n$ и взаимно простых с $n$.

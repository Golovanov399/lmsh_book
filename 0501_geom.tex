\begin{center}
\textbf{\Large Геометрические неравенства}\\
%\textit{Профи}\\
\textit{05.07.16}
\end{center}
\epigraph{\textit{По этим причинам Геральт был слабо знаком с местностью в Эмблонии или, в соответствии с более поздними картами, Понтарии и Приречья. Он не имел ни малейшего понятия, до какого из указанных на столбе населённых пунктов ближе и в какую сторону он должен двинуться с перекрестка, чтобы как можно скорее попрощаться с безлюдной пустошью и добраться до какой-либо цивилизации.}}{А. Сапковский, ``Сезон гроз''}

{\bf 1. Все должны знать.} \\
Пусть  $ABC$~--- треугольник. Тогда для его сторон справедливо неравенство $$AB+BC > AC >|AB-BC|.$$

В треугольнике напротив большей стороны лежит больший угол.

{\bf 2. Задача из теста.}\\
Докажите, что длина медианы меньше полусуммы прилегающих сторон.

\begin{problems}
 
\item В результате измерения четырёх сторон и одной из диагоналей некоторого четырехугольника получились числа: 1, 2, 2.8, 5, 7.5.
Чему равна длина измеренной диагонали?
% простая на нер-во треугольника, перебор

\item а) Докажите, что сумма диагоналей выпуклого четырехугольника меньше периметра.\\
б) Докажите, что сумма диагоналей выпуклого четырехугольника больше полупериметра.
% чуть сложнее, сложение неравенств

\item Внутри треугольника взяли две произвольные точки. Докажите, что расстояние между ними не превосходит наибольшей стороны треугольника.
%на применение предыдущей

\item Внутри треугольника взяли две произвольные точки. Докажите, что расстояние между ними не превосходит полупериметра треугольника. 
%дополнительное построение - 3

\item а) Точка $M$ расположена внутри треугольника $ABC$. Докажите, что \\ $~BM~+~CM~<~AB~+~AC$. \\
б) Докажите, что сумма расстояний от любой точки внутри треугольника до трех его вершин меньше периметра.
%дополнительное построение - 4

\item Четыре дома находятся в вершинах выпуклого четырехугольника. Где выкопать колодец, чтобы сумма расстояний от него до домов была минимальной? 
%на минимум простая

\item а) Коля и Вася живут по одну сторону от дороги. Вася хочет прийти к Коле в гости, купив по пути конфеты. Где нужно построить магазин, чтобы Вася прошел самое маленькое расстояние? (Магазин можно строить только у дороги).\\
б) Васин дом и школа находятся по разные стороны от большой дороги. Вася примерный мальчик и переходит дорогу только по пешеходному переходу (перпендикулярно дороге). В каком месте дороги нужно сделать пешеходный переход, чтобы Васин путь до школы был минимален?
%на минимум средняя

\item Полуостров, на котором живет Коля, представляет из себя острый угол. Коля хочет побывать у двух берегов и вернуться домой таким образом, чтобы длина пути была наименьшей. Как ему это сделать? (Коля ходит по прямой)
%на минимум сложная

\item На основании $AC$ равнобедренного треугольника $ABC$ отметили точку $D$, а на продолжении стороны $AC$ за точку $C$~--- точку $E$ таким образом, что $AD=CE$. Докажите, что $BD+BE>BA+BC$.

\item Точка $D$~--- середина основания $AC$ равнобедренного треугольника
$ABC$. Точка $E$~--- основание перпендикуляра, опущенного из точки $D$ на
сторону $BC$. Отрезки $AE$ и $BD$ пересекаются в точке $F$. Установите,
какой из отрезков $BF$ или $BE$ длиннее.

\item В выпуклом четырехугольнике $ABCD$ $\angle ABC= \angle BCD=120^{\circ}$. Докажите, что $AC+BD \geq AB+BC+CD $.

\item Расставьте на сторонах равностороннего треугольника $ABC$ точки $X$, $Y$ и $Z$ ($X$ на $BC$, $Y$ на $CA$ и $Z$ на $AB$) таким образом, чтобы периметр треугольника $XYZ$ был наименьшим.

\end{problems}
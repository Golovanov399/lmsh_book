\begin{center}
\textbf{\Large Легенды о короле Артуре}\\
\textit{22.07.16}
\end{center}

\epigraph{\it Всё выше, выше и выше\\
Стремим мы полёт наших птиц,\\
И в каждом пропеллере дышит\\
Спокойствие наших границ.}{Марш авиаторов}

\textbf{Упражнение 1.}\\
На доске написано несколько натуральных чисел. Каждую минуту выбирают какие-то два из них ($x$ и $y$) и заменяют их на числа $x-2$ и $y+1$. Докажите, что рано или поздно на доске появится отрицательное число.\\
\textbf{Упражнение 2.}\\
По кругу стоит 101 мудрец. Каждый из них либо считает, что Земля вращается вокруг Юпитера, либо считает, что Юпитер вращается вокруг Земли. Один раз в минуту все мудрецы одновременно оглашают свои мнения. Сразу после этого каждый мудрец, оба соседа которого думают иначе, чем он, меняет своё мнение, а остальные – не меняют. Докажите, что через некоторое время мнения перестанут меняться.

\begin{problems}

\item При дворе у короля Артура собралась свита из верных рыцарей. Рыцари дружны, однако в любом коллективе есть разногласия. Известно также, что любой рыцарь имеет не более трех врагов.  Король Артур собирается в поход, карать мятежных южных баронов. Докажите, что Артур сможет собрать карательный отряд и оставить остальных рыцарей править городом так, чтобы у каждый имел не более одного врага в своей группе?
% ну все и так помнят


\item Мерлин предложил королю Артуру, уставшему от походов, поиграть вот в такую игру: есть прямоугольная таблице $ m \times n$,  в ней записаны какие-то действительные числа. Разрешается менять знак сразу у всех чисел какой-либо строки или столбца. Докажите, что такими операциями можно добиться того, чтобы в каждой строке и в каждом столбце сумма чисел была неотрицательна.

\item Король Артур устроил пир по поводу перемирия, на который пригласил южных мятежных баронов. У каждого южного мятежного барона есть придворный шут, который хочет спеть песенку. Известно, что каждая песенка оскорбительна не более чем для
трех других южных мятежных Баронств. Каждый барон, дослушав своего шута, тут же уезжает и не слышит песенки, прозвучавшие после него. Докажите, что мудрый Мерлин может расположить выступления шутов в таком порядке, что каждый барон услышит не более трех оскорбительных песенок.

%сумма чисел во всей таблице увеличивается

%\item Из-за постоянных вооруженных разногласий по поводу политики, проводимой королем Артуром, его верным рыцарям некогда следить за своими землями. Управляющий заявил Ланселоту, что скоро участок рыцаря придет в негодность, ведь что уже 9 клеток его 100-клеточного квадратного поля поросли бурьяном. Известно, что бурьян за месяц распространяется на те и только те участки, у каждого из которых не менее двух соседних участков уже поражены бурьяном (участки соседние, если они имеют общую сторону). Докажите, что управляющий ошибается, и поле никогда не зарастет полностью. 
% граница бурьяна не возрастает

\item Король Артур постепенно завоевывает все новые и новые территории. В n окрестных графствах власть принадлежит либо королю, либо Нзависимому Южному Союзу Графов. Мятежные графы в большинстве своем очень трусливы. Каждый день в одном из графств( назовем его А) может поменяться власть. Это может произойти в том случае, если в большинстве граничащих с А графств власть принадлежит не тому, кто правит в графстве А(т.е., например,  если графство А примкнуло к Независимому Союзу, а большинство соседей присягнули Королю Артуру, и наоборот). Докажите, что смены правительств не могут продолжаться бесконечно. 
% После смены партии в стране количество ее стран-соседей той же партии увеличивается

\item  После карательного похода рыцари собрались за круглым столом и делят свою добычу. Известно лишь, что изначально у каждого рыцаря четное количество монет. По команде Мерлина каждый передает половину своих монет сидящему справа. Если после этого у кого-нибудь оказалось нечётное количество монет, то добрый южный барон, который не хочет, чтобы его повесили, добавляет этому рыцарю еще одну монету. Это повторяется много раз.  Доказать, что настанет время, когда у всех рыцарей будет поровну монет.

% Суммарная разность межу соседними дугами уменьшается 
%\item Даны две последовательности: 2, 4, 8, 16, 14, 10, 2 и 3, 6, 12. В каждой из них каждое число получено из предыдущего по одному и тому же закону.\\
%а) Найдите этот закон.\\
%б) Найдите все натуральные числа, переходящие сами в себя (по этому закону).\\
%в) Докажите, что число $2^{1991}$ после нескольких переходов станет однозначным.
%Следующее число -  удвоенная сумма цифр предыдущего. б - "Неподвижное число" - 18. в - Число не длится на 9 - значит, уменьшается
\item Темница короля Артура - длинный-длинный подземный коридор с камерами только по одну сторону, занумерованными числами от минус бесконечности до плюс бесконечности. В камерах находятся 9 узников (в одной камере могут быть заключены несколько узников). Их тюремщики скучают, и поэтому каждый день каких-то двух узников, обитающих в соседних камерах (k-й и (k+1)-й), переселяют соответственно в (k–1)-ю и (k+2)-ю комнаты. Докажите, что через конечное число дней тюремщики не смогут продолжать свою игру.(Узников, заключенных в одной камере, не расселяют.)

%\item В королевстве Артура магический турнир. На огромном поле соревнуются $n$ магов, никакие трое из них не стоят на одной прямой.По команде маги пускают фаерболы  в $n$ бесконечно длинных прямых заграждений(каждый - в свое), никакие два из заграждений не параллельны. Докажите, что Мерлин может так разбить магов на пары "маг - заграждение", что никакие два фаербола не столкнутся(Фаерболы интеллектуальные, и летят по наименьшему из возможных расстояний)
% Длины перпендикуляров уменьшаются
% Число пар врагов-соседей уменьшается
\item Прогуливаясь однажды по городскому базару инкогнито, Мерли увидел маленьких мошенников, играющих с горожанами в наперстки. Желая проучить мальчишек, Мерлин предложил им одно пари. Пари заключается в следуюшем: берется колода карт. Мальчишки переворачивают часть карт рубашкой вниз. Разрешается вынуть из колоды пачку из нескольких подряд идущих карт, в которой верхняя и нижняя карты лежат рубашкой вниз (в частности, можно вынуть просто одну карту рубашкой вниз), перевернуть эту пачку как одно целое и вставить в то же место колоды. Мерлин переворачивает ту пачку, на которую укажут мальчики (перевернуть что-то нужно каждым ходом). Мерлин выигрывает, если в колоде все карты лягут рубашкой вверх. Докажите, что в конце концов Мерлин выиграет.
% Полуинвариант: если закодировать буквами - в -вверх и н — вниз, то каждая операция улучшает лексикографический порядок

\end{problems}
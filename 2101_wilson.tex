\begin{center}
\textbf{\Large Теорема Вильсона}\\
%\textit{Профи}\\
\textit{21.07.16}
\end{center}

\epigraph{\it Теорема впервые была сформулирована Уорингом в 1770 году. Первое доказательство теоремы Вильсона было дано в 1771 году Лагранжем. Гаусс обобщил теорему Вильсона на случай составного модуля.}{Википедия}

\begin{problems}
\item Известно, что $x^2-1$ делится на простое число $p$. Какие остатки может давать $x$ при делении на $p$?

\item Дано простое число $p$  и его некоторый ненулевой остаток $a$.\\
\textbf{a)} Докажите, что существует и при том единственный остаток $b$, что $ab \mathop{\equiv}\limits_p 1$ (такой остаток $b$ называется \textit{обратным} остатка $a$).\\
\textbf{b)} Какие остатки совпадают со своими обратными остатками?

\item \textbf{(Теорема Вильсона)} Пусть $p$~--- некоторое простое число. Докажите, что $(p-1)! \mathop{\equiv}\limits_p -1$.
\item Докажите, что если $(n-1)! \mathop{\equiv}\limits_n -1$, то число $n$~--- простое.
 
\item Найдите обратные остатки:\\ 
\textbf{a)} $2$ и $3$ по модулю $11$; \\
\textbf{b)} $5$ по модулю $23$;\\
\textbf{c)} $17$ по модулю $37$.\\
\textbf{d)} Сопоставьте каждому остатку его обратный по модулю $7$. 

\item Пусть $p$~--- простое число. Докажите, что  $(p-k)! \cdot (k-1)! \mathop{\equiv}\limits_p (-1)^k$.

\item Пусть $p$~--- простое число. Докажите, что $(2p-1)!-p$ делится на $p^2$. 

\item Пусть числа $p$ и $p+2$ являются простыми числами-близнецами. Докажите, что справедливо   $4((p-1)!+1)+p\mathop{\equiv}\limits_{p^2+2p} 0$.

\item \textbf{a)} На доске написаны числа $\frac{100}{1}, \frac{99}{2}, \ldots, \frac{2}{99}, \frac{1}{100}$. Можно ли выбрать какие-то пять из них, произведение которых равняется единице? \\
\textbf{b)} Пусть произведение каких-то $2k+1$ чисел, написанных на доске, равно $\frac{m}{n}$. Докажите, что $m \mathop{\equiv}\limits_{101} -n$.  

\item Пусть $a_1, \ldots, a_p$~--- конечная арифметическая прогрессия с разницей, не кратной $p$. Докажите, что: \\
\textbf{a)} существует некоторый член последовательности $a_m$, делящийся на $p$;\\
\textbf{b)} существует некоторый член $a_k$, такой что $a_k+a_1\cdot a_2\cdot\ldots\cdot a_p$ делится на $p$;\\
\textbf{c)} существует некоторый член $a_k$, такой что $a_k+a_1\cdot a_2\cdot\ldots\cdot a_p$ делится на $p^2$.

\item 
\textbf{a)} Найдите все простые числа $p$, такие что $(p-2)!$ не делится на $(p-1)$.\\
\textbf{b)} Дано простое число $p$. При каких $n$ число $p^n-1$ делится на $(p-1)^2$?\\
\textbf{c)} Для каких натуральных $n$ число $(n-1)!+1$ является точной степенью $n$?


\item Докажите, что существует бесконечно много таких пар различных натуральных чисел 
$k,n>1$, что $(k!+1,n!+1)>1$; 


\end{problems}

\begin{center}
\textbf{\Large Перекладывание отрезков}\\
\textit{16.07.16}
\end{center}

\epigraph{\it Выживает только сильнейший.}{Ч. Дарвин}

\begin{problems}
\item На катетах $AC$ и $BC$ равнобедренного прямоугольного треугольника отметили точки $M$ и $L$ 
соответственно так, что $MC=BL$. Точка $K$ --- середина гипотенузы $AB$. Докажите, что треугольник 
$MKL$ также является прямоугольным равнобедренным. 

\item В треугольнике $ABC$ биссектриса $AE$ равна по длине отрезку $EC$. Причем $2AB=AC$. Найдите 
углы треугольника $ABC$.

\item В равнобедренном прямоугольном треугольнике $ABC$ на гипотенузе $AB$ взяты точки $M$ и $N$ ($N$ между $M$ и $B$) такие, что $\angle MCN=45^{\circ}$. Докажите, что из отрезков $MN, AM, NB$ можно составить прямоугольный треугольник.

\item В равнобедренном треугольнике $ABC\; (AB = BC)$ на боковую сторону $BC$ опущена высота $AH$. Точка $L$~--- основание перпендикуляра из $H$ на сторону $AB$. Оказалось, что $AL = AB/4$. Найдите углы треугольника $ABC$.
%легко, счет углов

\item На боковых сторонах $AB$ и $AC$ равнобедренного треугольника $ABC$ отметили соответственно точки $K$ и $L$ так, что $AK = CL$ и $\angle ALK + \angle LKB = 60^{\circ}$. Докажите, что $KL = BC$.

\item На гипотенузе~$AC$ прямоугольного треугольника $ABC$ выбрали точку~$D$ такую, что $BC=CD$. На катете~$BC$ выбрали такую точку~$E$, что $DE=CE$. Докажите, что $AD+BE=DE$. 
% На равенство треугольников с доп построением

\item В квадрате $ABCD$ точки $K$ и $M$ принадлежат сторонам $BC$ и $CD$  соответственно, причем $AM$~--- биссектриса угла $\angle KAD$. Докажите, что $AK=DM+BK$.

%\item В равнобедренном треугольнике $ABC$ с основанием $AC$ проведена биссектриса $AD$. Известно, что $AD+BD = AC$. Найдите углы треугольника.
%трудновато, надо правильно отложить пару отрезков и потом считать углы

\item На медиану $BM$ треугольника $ABC$ опустили перпендикуляр $AL$ и перпендикуляр $DK$ из некоторой точки $D$ на стороне $AB$ ($L$ и $K$~--- различные точки, лежащие внутри $BM$). Оказалось, что $BK = LM$. Докажите, что $CD = BD+BA$.
%удвоение медианы, внимательно смотреть, не очень легко

\end{problems}
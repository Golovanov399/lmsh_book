\begin{center}
\textbf{\Large Индукция в графах}\\
\textit{19.07.16}
\end{center}

\epigraph{\it ...и, когда он уснул, взял одно из ребр его...}{Книга Бытия}

Разбор. Докажите по индукции, что в дереве с $n$ вершинами $n-1$ ребро.

\begin{problems}

\item
Усадьбы любых двух джентльменов в графстве Вишкиль соединены либо водным (лодочка), либо сухопутным (карета) сообщением. Докажите, что можно закрыть один из видов транспорта так, чтобы любой джентльмен мог по-прежнему добраться до любого другого.

\item В одном государстве 100 городов, и каждый соединён с каждым дорогой с односторонним движением. Докажите, что можно поменять направление движения не более, чем на одной дороге, так, чтобы от каждого города можно было доехать до любого другого.

\item В графе степень каждой вершины не превосходит $d$. Докажите, что все вершины графа можно покрасить в $d + 1$ цвет так, чтобы любые две вершины, соединённые ребром, имели разный цвет. 

%\item (вершины, средняя)
\item  а) Докажите, что в турнире (полном ориентированном графе) найдётся путь, который проходит по каждой вершине ровно один раз. \\
б) В шахматном турнире каждый сыграл с каждым по партии. Докажите, что участников можно перенумеровать так, что участник с номером $n$ не проиграл участнику с номером $n+1$.

\item Докажите, что граф двудольный, если в нем нет циклов нечётной длины, используя:\\
а) индукцию по количеству вершин;\\
б) индукцию по количеству ребер.

%\item (рёбра) формула Эйлера

\item На конференцию приехали учёные, каждый из них знает несколько языков. Оказалось, что любые трое человек могут поговорить друг с другом (возможно, одному из них придётся переводить для остальных). Докажите, что учёные могут разбиться на пары, говорящие на одном языке, если общее число учёных чётное.

\item В стране 100 городов и несколько дорог. Каждая дорога соединяет два каких-то города. Из каждого города можно добраться до любого другого, двигаясь по дорогам. Докажите, что можно объявить несколько дорог главными так, чтобы из каждого города выходило нечётное число главных дорог.

%дальше могут быть драконы
%\item (Вершины) Школьники играли в настольный теннис "на победителя". Они установили очередь и правила: вначале играют первый и второй, а в дальнейшем каждый очередной участник играет с победителем предыдущей пары. На следующий день те же школьники снова сыграли по тем же правилам, но очередь шла в обратном порядке (вчерашний последний стал первым, предпоследний – вторым, и так далее). Известно, что каждый сыграл хотя бы раз и в первый день, и во второй. Докажите, что найдутся два школьника, которые играли между собой и в первый день, и во второй.

%\item (вершины, конструктив, забава)$n$ человек не знакомы между собой. Нужно так познакомить друг с другом некоторых из них, чтобы ни у каких трех людей не оказалось одинакового числа знакомых. Докажите, что это можно сделать при любом $n$.

%\item (не добро, вершинки по три) Пусть в графе не менее чем  $3n - 2$  вершины и не более чем  $3n - 2$  ребра  ($n \geq 2$).  Тогда найдутся $n$ вершин, между которыми нет ребер. 

%\item (рёбра??) В стране некоторые пары городов соединены дорогами. Оказалось, что нет трёх городов, соединённых дорогами попарно. Кроме того, для любых $n$ дорог, найдётся город, из которого выходит хотя бы две из них. Докажите, что города можно так разбить на $n$ округов, что дорог внутри округов не будет.

%\item В компании из  $2n + 1$ человека для любых $n$ человек найдется отличный от них человек, знакомый с каждым из них. Докажите, что в этой компании есть человек, знающий всех.



\end{problems}
\begin{center}
\textbf{\Large   Алгоритм Евклида }\\
%\textit{Профи}\\
\textit{15.07.16}
\end{center}

\epigraph{\it То, что принято без доказательства, может быть отвергнуто без доказательства.}{Евклид}

\indent \textbf{Определение 1.} \textit{Наибольшим общим делителем} двух целых чисел называется такой их общий делитель, который больше всех остальных общих делителей.\\
\indent \textbf{Определение 2.} \textit{Наименьшим общим кратным} двух целых чисел называется наименьшее положительное число, делящееся на оба этих числа.\\
\textbf{Обозначение.} НОД$(a,b)=(a,b)$, НОК($a,b$)=[$a,b$].\\
\textbf{Определение 3.} Если $(a,b)=1$, то числа $a$ и $b$ называются \textit{взаимно простыми}. 
\textbf{Алгоритм Евклида.}\\
Пусть $a$ и $b$ -- целые числа, не равные одновременно нулю, и последовательность чисел
$a > b > r_1 > r_2 > r_3 > r_4 > ... > r_n$
определена тем, что каждое $r_k$~-- это остаток от деления предпредыдущего числа на предыдущее, а предпоследнее делится на последнее нацело, то есть \\
\begin{equation*}
 a = b \cdot q_0 + r_1
\end{equation*}
\begin{equation*}
b = r_1  \cdot q_1 + r_2
\end{equation*}
\begin{equation*}
r_1 = r_2  \cdot q_2 + r_3
\end{equation*}
\begin{center} $\cdots$ \end{center}  
\begin{equation*}
r_{k-2} = r_{k-1}  \cdot q_{k-1} + r_k
\end{equation*}
\begin{center} $\cdots$ \end{center}
\begin{equation*}
r_{n-2} = r_{n-1}  \cdot q_{n-1}+ r_n
\end{equation*}
\begin{equation*}
r_{n-1} = r_n  \cdot q_n
\end{equation*}
Тогда $(a,b)=r_n$.\\
\textbf{Упражнение. } Найдите \textbf{a)} ($1333$, $372$);  \textbf{b)} ($561$, $4301$);   \textbf{c)} ($9163$, $3311$).



\begin{problems}
\item Докажите следующий свойства НОД и НОК:\\
\textbf{a)} если $m$ делится на $n$, то $(m$, $n)=n$, $[m$, $n]=m$;\\
\textbf{b)} $(a \cdot m$, $a \cdot n$)=$a\cdot (m$, $n)$;\\
\textbf{c)} если $(m$, $n)=d$, то $(\frac{m}{d}$, $\frac{n}{d})=1$;\\
\textbf{d)} для любого целого числа $a$ справедливо, что $(m$, $n$)=($m-a \cdot n$, $n)$;\\
\textbf{e)} $(m$, $n) \cdot [m$, $n]=m\cdot n$.
\item \textbf{a)} Докажите, что ($n+3$, $5n+14$)  взаимно просты при любом целом $n$.\\  
\textbf{b)} Какие значения может принимать ($3n+2$, $10n+23$).
\item Вася посчитал НОК всех чисел от $1$ до $1000$, а Петя -- всех чисел от $501$ до $1000$. У кого результат получился больше и во сколько раз?

\item Пусть $m,n$ -- целые взаимнопростые числа. Найдите наибольшее возможное значение $(n+2016m$, $m+2016n)$.

\item Пусть $a$ и $b$ -- натуральные числа. Докажите, что среди чисел $a, 2a, 3a,...,ba$ ровно ($a$, $b$) чисел делится на $b$.

\item Изначально на доске записаны числа $m$ и $n$. Каждую минуту Саша записывает в тетрадку квадрат наименьшего из чисел на доске, после чего Даша ищет разность чисел на доске и записывает ее вместо наибольшего из них, пока в какой-то момент не выпишет $0$. Чему равна сумма  чисел у Саши в тетради?

\item Пусть $a$ и $b$ –- взаимно простые натуральные числа. В доме есть лифт с двумя кнопками, одна из которых
поднимает лифт на $a$ этажей вверх, а вторая опускает на $b$ этажей вниз, если это возможно (например, на последнем этаже первая кнопка не работает). Докажите, что на этом лифте можно попасть с любого этажа на любой другой, если высота дома не меньше
\textbf{а)} $2ab$; \textbf{б)} $a + b$.

%\item Натуральные числа $a, b$ и  $c$ удовлетворяют равенству:
%\begin{equation*}
%[a,c]-[b,c]= a-b
%\end{equation*}
%Докажите, что $a$, $b$ делятся на $c$. 
%Докажите, что в вершинах любого графа можно расставить натуральные числа так, что любые два числа,
%соединенные ребром имеют НОД>1, а числа, не соединенные ребром, взаимно просты.

\end{problems}

\begin{center}
\textbf{Перегородка}
\end{center}

\begin{problems}
\item На доске написаны взаимнопростые числа $m$ и $n$. Каждую минуту модуль разности этих чисел записывается вместо наибольшего из них. Докажите, что в какой-то момент на доске будут написаны две единицы. 
\item Пусть $p$~--- простое число. Сколько существует пар взаимнопростых натуральных чисел $(m,n)$ таких, что $p=m+n$? 
\item Аня нашла себе интересное занятие. Она написала на доске две единички, потом между ними написала их сумму. Ее это так захватило, что она продолжила: брала ряд чисел, который у нее получился на предыдущем шаге, и между двумя соседними числами писала их сумму (старые числа при этом не стирала).\\ \textbf{a)} Сколько раз она выписала простое число $p$?\\  \textbf{b)} Сколько раз она выписала произвольное число $n$? 

\end{problems}

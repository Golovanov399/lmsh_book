\begin{center}
\textbf{\Large Инвариант}\\
%\textit{Обычные группы}\\
\textit{06.07.16}
\end{center}

\epigraph{\it А вы, друзья, как ни садитесь,\\
Всё в музыканты не годитесь.}{И. А. Крылов, ``Квартет''}

{\bf Задачи на разбор.}
\begin{problems}
\item На шести елках сидят шесть чижей, на каждой елке - по чижу. Елки растут в ряд с интервалами в 10 метров. Если какой-то чиж перелетает с одной елки на другую, то какой-то другой чиж обязательно перелетает на столько же метров, но в обратном направлении. Могут ли все чижи собраться на одной елке? 

	%\item В ряд выстроены 100 фишек. Разрешено менять местами две фишки, стоящие через одну фишку. Можно ли с помощью таких операций переставить все фишки в обратном порядке? 
	
  %  \item Утром в луже плавало 19 синих и 95 красных марсианских амеб. Иногда они сливались: если сливаются две красные, то получается одна синяя амеба, если сливаются две синие, то получившаяся амеба тут же делится и в итоге образуются четыре красные амебы, наконец, если сливаются красная и синяя амеба, то это приводит к появлению трех красных амеб. Вечером в луже оказалось 100 амеб. Сколько среди них синих?
    
    \item На столе стоят 16 стаканов. Из них 15 стаканов стоят правильно, а один перевернут донышком вверх. Разрешается одновременно переворачивать любые четыре стакана. Можно ли, повторяя эту операцию, поставить все стаканы правильно?
        
    \item В пробирке находятся марсианские амебы трех типов:$A$, $B$ и $C$. Две амебы любых двух разных типов могут слиться в одну амебу третьего типа. После нескольких таких слияний в пробирке оказалась одна амеба. Каков ее тип, если исходно амеб типа $A$ было $20$ штук, типа $B - 21$ штука и типа $C - 22$ штуки?
\end{problems}
\resetproblem
{\bf Задачи для самостоятльного решения.}
\begin{problems}

\item На площадке возле девятого корпуса шишками выложены три числа 1000, 1111 и 2016. Каждую минуту Аня заменяет
    
а) имеющиеся три числа $a$, $b$, $c$ на числа $\frac{a+b}{2}, \frac{a+c}{2}, \frac{b+c}{2}.$

б) произвольные два числа $a$ и $b$ на $\frac{a^2}{b}, \frac{b^2}{a}$

в) произвольные два числа $a$ и $b$ на $\frac{a+b}{\sqrt{2}}, \frac{a-b}{\sqrt{2}}$

Могут ли получиться числа $2015,~2016,~2017$?

\item  На сосне растут 8 бананов и 7 апельсинов. Если сорвать два одинаковых фрукта, то на сосне тут же вырастет один банан, а если сорвать два разных – вырастет один апельсин. Срывать фрукты по одному нельзя. В конце концов на сосне остался один фрукт. Какой?

\item
Клетки доски $10 \times 10$ покрашены в белый цвет. За один ход разрешается перекрасить все клетки квадрата $6 \times 6$ в противоположный цвет. Можно ли за конечное число ходов получить шахматную раскраску доски?

\item Круг разделен на 6 секторов. Разрешается добавлять по одному камешку в любые два соседних сектора. Можно ли добиться, чтобы во всех секторах было поровну камешков, если в начале в двух секторах, расположенных через один, лежит по камешку
     
    
\item На столе лежит куча из 2017 камней. Ход состоит в том, что из какой-либо кучи, содержащей более одного камня, выкидывают камень, а затем одну из куч делят на две. Можно ли через несколько ходов оставить на столе только кучки, состоящие из трех камней? 

\item
В квадрате $n \times n$ верхний правый угол покаршен в белый цвет, а все остальные клетки в чёрный. Разрешается в столбце или строке перекрасить все клетки в противоположный цвет. Можно ли добиться того, что весь квадрат будет белый?
    
\item В центре каждой клетки шахматной доски стоит по фишке. Фишки переставили так, что попарные расстояния между ними не уменьшились. Докажите, что в действительности попарные расстояния не изменились.
    
	%\item Есть два стакана - в одном концентрированный вишкильский компот, в другом - раствор концентрированного вишкильского компота. Полную чайную ложку компота из первого стакана переливают во второй, размешивают, потом переливают ложку из второго в первый. 
    
    %а) В каком стакане компот более высокой концентрации?
    
    
    %б) А если описанную манипуляцию повторить тысячу раз?
 
    \item Камни лежат в трёх кучках: в одной – 51 камень, в другой – 49 камней, а в третьей – 5 камней. Разрешается объединять любые кучки в одну, а также разделять кучку из чётного количества камней на две равные. Можно ли получить 105 кучек по одному камню в каждой? 
      
	\item На доске были записаны числа 2, 5 и 8. Разрешалось сложить два записанных числа, вычесть из этой суммы третье, а результат записать на доску вместо того числа, которое вычиталось. После многократного выполнения такой операции на доске оказались три числа, наименьшее из которых равно 2016. Найдите остальные числа.

\end{problems}
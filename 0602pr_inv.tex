\begin{center}
\Large
\textbf{Инвариант и не только. 06 июля}
\end{center}


\bigskip
\large
{\it Часть первая.}

\bigskip

\q1. Все клетки доски $8\times 8$ окрашены в белый цвет. Петя перекрасил одну клетку в чёрный. Вася может поменять цвет всех клеток одной строки или столбца. Сможет ли Вася снова сделать всю доску белой?

\q2. На доске написаны числа $1, 5, 10$. Каждую минуту можно любое из чисел заменить на разность суммы двух других и этого числа. Можно ли через некоторое время получить на доске тройку $2013, 2016, 2023$?

\q3. За один ход можно заменить упорядоченную тройку целых чисел
$(p, q, r)$ на тройку $(r+5q, 3r-5p, 2q-3p)$. Существует ли целое число $k$, 
для которого из тройки $(1, 6, 7)$ можно за конечное число шагов получить 
тройку $(k, k+1, k+2)$? 

\q4. В квадрате $4\times 4$ одна крайняя неугловая клетка закрашена в чёрный цвет, а остальные клетки белые. За одну операцию разрешается поменять цвета всех клеток одной линии (будь то строка, столбец или диагональ). Удастся ли сделать доску одноцветной?

\q5. По кругу растут $n$ деревьев. На каждом дереве сидит сова. Каждую минуту какие-то две совы перелетают на соседние деревья, причём одна из них по часовой стрелке, а другая --- против часовой. При каких $n$ все совы смогут оказаться на одном дереве? 



\bigskip

{\it Часть вторая.}

\bigskip

\q6. На доске написаны числа от $1, 2, \ldots 100$. Вася каждую минуту стирает два числа $a$ и $b$ и заменяет их на а)~$a+b$; б)~$ab$; в) $\sqrt{a^2+b^2}$; г) $\frac{ab-1}{a+b+2}$. Докажите, что число, которое останется последним, не зависит от порядка действий.

\q7. С написанными на доске положительными числами разрешается выполнить одну из двух следующих операций:
1)~стереть произвольное число~$x$ и записать два раза число  $\sqrt{x+1}-1$;
2)~стереть два произвольных числа~$x$ и~$y$ и записать число~$x+y+xy$.
Изначально на доске написано число~$a$.
Через несколько операций на доске оказалось написано одно число. Докажите, что оно равно $a$.


\q8. Петя выписал на доске числа от 1 до 100. Каждую минуту Вася стирает с доски два числа, пишет на доску их сумму, а себе в блокнот пишет их произведение. \\
а) Докажите, что сумма чисел  на блокноте после 99 операций не зависит от порядка Васиных действий.\\
б) Чему равна эта сумма?


\q9 . В квадрате $10\times10$ расставлены числа от~1 до~100 следующим образом: в первой строке (слева направо 
по порядку) 1, 2,~\dots 10, во второй~--- 11, 12,~\dots 20,~\dots, в десятой~--- 91, 92,~\dots 100. Разрешается 
взять любой прямоугольник $1\times3$ и сделать следующую операцию: прибавить к крайним числам по~1, а из 
среднего отнять~2, или сделать обратную операцию. Через некоторое время оказалось, что в квадрате опять 
присутствуют все числа от~1 до~100. Докажите, что они расположены на первоначальных местах.
\begin{center}
\textbf{\Large Ортоцентр треугольника}\\
\textit{20.07.16}
\end{center}

\epigraph{\it Прожита медиана,\\
Прожита биссектриса,\\
А золотое сечение\\
Прожито даже дважды.}{Деление на ноль~--- 13000}

\begin{problems}
\item Дан прямоугольный треугольник. Впишите в него прямоугольник с общим прямым углом, у которого диагональ минимальна.

\item Пусть $M$~--- основание перпендикуляра, опущенного из вершины $D$ параллелограмма $ABCD$ на диагональ $AC$. Докажите, что перпендикуляры к прямым $AB$ и $BC$, проведённые через точки $A$ и $C$ соответственно, пересекутся на прямой $DM$.

\item В треугольнике $PQR$  $\angle QPR=46^{\circ}$. Через вершины $P$ и $R$ проведены перпендикуляры к сторонам $QR$ и $PQ$ соответственно. Точка пересечения этих перпендикуляров находится от вершин $P$ и $Q$ на расстоянии, равном $1$. Найдите углы треугольника $PQR$.

\item В прямоугольнике $ABCD$ биссектрисы угла $B$ и внешнего угла $D$ пересекают сторону $AD$ и прямую $AB$ в точках $K, M$ соответственно. Докажите что отрезок $KM$ равен и перпендикулярен отрезку $BD$.

\item В треугольнике $ABC$ сторона $AC$ наименьшая. На сторонах $AB$ и $CB$ взяты точки $K$ и $L$ соответственно, причём  $KA = AC = CL$.  Пусть $M$~--- точка пересечения $AL$ и $KC$, а $I$~--- точка пересечения биссектрис треугольника $ABC$. Докажите, что прямая $MI$ перпендикулярна прямой $AC$.

\item а) \textbf{(Параллелограмм Вариньона)} Докажите, что в произвольном четырехугольнике середины сторон образуют параллелограмм.\\
б) В четырехугольнике три угла равны $45^{\circ}$. Докажите, что параллелограмм Вариньона~--- квадрат.

\item Диагонали выпуклого четырехугольника $ABCD$ взаимно перпендикулярны. Через середины сторон $AB$ и $AD$ проведены прямые, перпендикулярные противоположным сторонам $CD$ и $CB$ соответственно. Докажите, что эти прямые и прямая $AC$ имеют общую точку.

\item В прямоугольнике $ABCD$ точка $M$~--- середина стороны $CD$. Через точку $C$ провели прямую, перпендикулярную прямой $BM$, а через точку $M$~--- прямую, перпендикулярную диагонали $BD$. Докажите, что два проведенных перпендикуляра пересекаются на прямой $AD$.

%\item Дан прямоугольник $ABCD$ и точка $P$. Прямые, проходящие через $A$ и $B$ и перпендикулярные, соответственно, $PC$ и $PD$, пересекаются в точке $Q$. Докажите, что $PQ \perp AB$.

%\item В параллелограмме $ABCD$ опустили перпендикуляр $BH$ на сторону $AD$. На отрезке $BH$ отметили точку $M$, равноудалённую от точек $C$ и $D$. Пусть точка $K$~--- середина стороны $AB$. Докажите, что угол $MKD$ прямой.
\end{problems}

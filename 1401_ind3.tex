\begin{center}
\textbf{\Large Индукция-2}\\
\textit{14.07.16}
\end{center}

\epigraph{\it Элемент тока длины $dl$ создаёт поле с магнитной индукцией $dB = k\frac{Idl}{r^2}$.}{Закон Био-Савара-Лапласа}

{\bf Задачи на разбор}
\begin{problems}

\item Докажите, что $3^{2n+2}+8n-9$ делится на 16.
\item Докажите, что любое число можно представить в виде суммы степеней двоек единственным образом.

%\q6.
%{\itshape Утверждение.} У всех людей глаза одинакового цвета. \\
%{\itshape Доказательство. } Докажем, что в компании из $n$  людей у всех глаза одинакового цвета, методом
%математической индукции.
%При     $n=1$    утверждение очевидно. \\

%Предположим, что в любой компании из $k$  человек у всех глаза одинакового цвета. Рассмотрим ГШй йюж
%компанию из   $k+1$   человека. Перенумеруем их: первый, второй, третий, …, $k$ -ый,    $(k+1)$ -ый. По
%предположению, у любых   из этих людей глаза одинакового цвета. Значит, у первого, второго,
%…,  $k$-го — глаза одинакового цвета. \\

%Возьмём теперь другую группу из человек: второй, третий, …,  $k$-ый, $(k+1)$-ый. У них тоже
%глаза одинакового цвета. В обеих группах присутствовал второй человек. Значит у всех них глаза
%того же цвета, как и у второго, то есть у всех одинаковые. {\bf Где же ошибка?}
\end{problems}
\resetproblem


{\bf Задачи для самостоятельного решения}
\begin{problems}

%\item
%Докажите, что $$1^3+2^3+3^3+ \ldots + n^3 = %\dfrac{n^2(n+1)^2}{4}$$


%\item
%Докажите, что для любого $x \ne 1$ $$1 + x + x^2 + %\ldots + x^n =
%\frac{x^{n+1} - 1}{x - 1}$$
\item
Докажите равенство $$3 + 33 + 333 + \ldots + \underbrace{33 \ldots
33}_n = \dfrac{10^{n+1} - 9n - 10}{27}$$
%\item
%Докажите, что
%$$ (1 - \frac{1}{4}) (1 - \frac{1}{9}) \ldots (1 - \frac{1}{n^2}) = \frac{n+1}{2n}$$

\item Числа вида $F_n = 2^{2^n} + 1$ называются числами Ферма. Докажите, что десятичная запись числа $F_n$ при $n \geqslant 2$ оканчивается цифрой 7.

\item Докажите, что все числа 10017, 100117, 1001117, \ldots делятся на 53.
\item Кусок бумаги разрешается рвать на 4 или на 6 кусков. Докажите, что по этим правилам его можно разорвать на любое число кусков, начиная с шести.
%Докажите, что любой квадрат можно разрезать на $n$ квадратов, начиная с шести.
\item На столе стоят $2^n$ стаканов с водой. Разрешается взять любые два стакана и уравнять в них количества воды, перелив часть воды из одного стакана в другой. Докажите, что с помощью таких операций можно добиться того, чтобы во всех стаканах было поровну воды. 
\item Докажите, что любое число можно представить в виде суммы нескольких различных чисел Фибоначчи. Единственно ли такое представление?
\item На кольцевой дороге стоят $n$ машин. При этом, суммарное количество топлива у машин  равно количнству топлива чтобы проехать один круг. Докажите, что существует машина, которая может проехать весь круг, забирая топливо у других машин.

\strut\hrule

\item Докажите, что $3^{2^n} - 1$ делится на $2^{n+2}$, но не делится на $2^{n+3}$.
\item Докажите,  что любое количество квадратов можно разрезать на несколько частей, из которых можно состваить квадрат.
%сложные
%\item
%Пусть $p$ -- простое и $gcd(n,p) = 1$. Докажите, что $n^{p-1}$ даёт остаток 1 при делении на $p$.
%\item На сколько частей делят пространство $n$ плоскостей,если никакие три из которых не пересекаются по одной прямой и никакие 4 из них не имеют общей точки? (на какое наибольшее число частей могут разбивать плоскость $n$ окружностей)
%делимость
%\item Докажите, что $5 \cdot 2^{3n+1} + 3^{3n+2}$ делится на 19.
%\item Докажите, что $2^{3^n}$ + 1 делится на $3^{n+1}$.
\item Даны два взаимнопростых числа $m$ и $n$, а так же число 0. Имеется калькулятор, который умеет выполнять лишь одну операцию: вычислять среднее арифметическое двух чисел одной чётности. Докажите, что с помощью этого калькулятора можно получить все числа от $1$ до $n$.
\item  На окружности взяли $n$ точек и соединили их всевозможными отрезками. Оказалось, что никакие три не пересекаются в одной точке. На сколько частей они делят круг?
\end{problems}
\begin{center}
\textbf{\Large Деревья}\\
%\textit{Профи}\\
\textit{10.07.16}
\end{center}

\epigraph{\it Дуров, верни стену!}{Довольные клиенты ВКонтакте после введения микроблога.}

%\textbf{Определение.} Последовательность вершин $v_1, v_2, \dots, v_k$ называется путем, если в графе есть ребра $v_1v_2, v_2v_3, \dots, v_{k - 1}v_k$. Путь называется простым, если все его вершины попарно различны.\\
%\textbf{Определение.} Путь называется циклом, если $v_1 = v_k$. Цикл называется простым, если вершины $v_1, v_2, \dots, v_{k - 1}$ попарно различны.\\
\textbf{Определение.} Будем давать несколько определений дерева:\\
1. Связный граф без циклов.\\
%отсутствие простых циклов и циклов суть одно и то же
2. Граф, между любыми двумя вершинами которого существует единственный путь.\\
3. Связный граф, который при удалении любого ребра перестает быть связным.\\
4. Связный граф, количество ребер которого на 1 меньше количества вершин.\\
\textbf{Определение.} Листом (висячей вершиной) называется вершина дерева, которая имеет степень 1.\\
\begin{center}
\textbf{Теоретические задачи}\\
\end{center}
\begin{problems}
\item Докажите, что первое и второе определения дерева эквивалентны.
\item Докажите, что первое и третье определения дерева эквивалентны. 
\item а) Докажите, что в каждом дереве (по первому определению) из более чем одной вершины есть лист.\\
%двудольность дерева
б) Докажите, что в каждом дереве (по первому определению) из более чем одной вершины есть хотя бы два листа.
\item Докажите, что первое и четвертое определения дерева эквиваленты.
%взять граф на n вершинах, добавить ребро, спросить где подвох 
%об индукции
\item \textbf{Лемма о существовании остовного дерева (скелета).} Докажите, что из каждого связного графа можно удалить некоторое число ребер так, чтобы получилось дерево. 
%через первое определение
%остовное дерево не единственно
%Докажите, что если в графе нет циклов, то в него можно добавить ребер так, чтобы получилось дерево.

\end{problems}
\begin{center}
\textbf{Практические задачи}\\
\end{center}
\resetproblem

\begin{problems}
\item Существует ли граф, у которого есть два остовных дерева без общих ребер?
\item Существует ли дерево на 9 вершинах, в котором 2 вершины имеют степень 5?
\item Невод браконьера представляет собой прямоугольную сетку $100\times 100$ клеток. После каждой поимки инспектор рыбоохраны обрезает в неводе одну веревочку (указанную браконьером), так, чтобы невод не распался на части. Сколько задержаний может выдержать браконьер до разрушения своего инструмента?
\item Имеется связный граф. Докажите, что в нем можно выбрать одну из вершин так, чтобы после ее
удаления вместе со всеми ведущими из нее ребрами останется связный граф.
\item В доску вбито 2016 гвоздей. Двое играют в игру, делая ходы по очереди. За ход можно соединить два еще не соединенных  между собой гвоздя ниткой. Кто выиграет при правильной игре, если получивший замкнутую цепь
    
    а)~проигрывает;
    
    б)~выигрывает?
\item В стране 100 городов, некоторые из которых соединены авиалиниями. Известно, что от любого города можно долететь до любого другого (возможно, с пересадками). Докажите, что можно побывать в каждом городе, совершив не более  а) 198 перёлетов;  б) 196 перелётов. 
\end{problems}
\begin{center}
\textbf{Стена}\\
\end{center}
\begin{problems}
\item а) В дереве нет вершин степени 2. Докажите, что количество листов
больше половины общего количества вершин.\\
б) Пусть $d$~--- наибольшая степень вершины дерева. Докажите, что в этом дереве есть хотя бы $d$ листов.
\item В графе на 31 вершине все ребра покрашены в один из трех цветов так, что при удалении всех ребер любого цвета граф остается связным. Какое минимальное количество ребер может быть в этом графе?
\item В стране Летняя есть $n$ городов, которые соединены дорогами так, что из каждого города можно добраться до каждого. У каждой дороги есть прочность (у всех дорог разная), то есть максимальный вес автомобиля, который может по ней проехать. Каждый житель страны знает прочность каждой дороги, но не вес собственного автомобиля, потому планирует свое путешествие из одного города в другой так, чтобы минимальная прочность дороги, по которой он проедет, была максимальна. Правительство заметило, что в таком случае некоторыми дорогами никогда не воспользуется ни один житель, а потому было решено снести все неиспользуемые дороги и посадить на их месте кукурузу. Сколько дорог осталось в стране Летняя?
\end{problems}
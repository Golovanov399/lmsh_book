\begin{center}
\textbf{\Large Вступительная олимпиада}\\
%\textit{Профи}\\
\textit{04.07.16}
\end{center}


\begin{problems}

\item На острове Мадагаскар есть Холм и Озеро. Глория идет от Холма к Озеру, а навстречу ей от Озера бежит Алекс. Известно, что Глория проходит этот путь за 5 часов, а Алекс всего за час. Через 50 минут после встречи Глории и Алекса Марти также прошел от Озера к Холму, причем это расстояние Марти проходил 1 час 40 минут. Через сколько минут после встречи с Глорией Марти дойдет до Холма, если Глория и Алекс начали движение одновременно? Ответ обоснуйте.

%\item Вася и Петя получили от своих родителей по 100 рублей и решили покататься по городу. Вася катался на маршрутках за 17 и за 10 рублей, а Петя~--- на автобусах за 12 рублей. К вечеру оказалось, что они поездили одинаковое количество раз и потратили одинаковое количество денег. Сколько у них осталось?

%\item На главной диагонали шашечной доски $10\times 10$ стоит 10 шашек (все в разных клетках). За один ход разрешается взять любую пару шашек и передвинуть каждую из них на одну клетку вниз. Можно ли за несколько ходов поставить все шашки на нижнюю горизонталь?

%\item На столе в виде треугольника выложены 28 монет одинакового размера (рис.). Известно, что суммарная масса любой тройки монет, которые попарно касаются друг друга, равна 10  г. Найдите суммарную массу всех 18  монет на границе треугольника.
%\begin{figure}[h!]
%\center{\includegraphics[scale=0.5]{coins.png}}
%\end{figure}\\

\item Натуральные числа от 1 до 20 расставили по кругу в некотором порядке, а затем покрасили в красный цвет те из них, которые являются делителями своего правого соседа. Какое наибольшее количество красных чисел могло получиться?

\item В выпуклом пятиугольнике $ABCDE$ углы $ABC$ и $CDE$ равны, $AB=ED$, $BC=CD$. Докажите, что отрезки $AD$ и $BE$ равны.

\item Пусть $d_1, d_2 , d_3$ и $d_4$~--- наименьшие различные делители натурального числа $n$. Оказалось, что $d_1^2+d_2^2+d_3^2+d_4^2=n$. Чему могло быть равно $n$ (укажите все варианты)?

\item Если в детектор фальшивых монет опустить 5 монет весом $a, b, c, d, e$ граммов, где $a<b<c<d<e$, то он сбросит монеты весом $b$ и $c$ граммов в правую чашу, а остальные в левую. Есть 50 монет попарно различных по весу, они пронумерованы и легко различаются по внешнему виду. Как при помощи детектора определить самую легкую монету?

\item На острове рыцарей (которые говорят только правду) и лжецов (которые всегда
лгут) состоялся шахматный фестиваль. 64 любителя шахмат встали по
одному на клетки большой шахматной доски. После этого каждый сказал:
``{\it Среди людей, стоящих со мной на одной горизонтали, больше лжецов, чем
среди людей, стоящих со мной на одной вертикали}''. Докажите, что
количество рыцарей делится на~8.

\end{problems}

\begin{center}
\textbf{\Large Малая теорема Ферма}\\
%\textit{Профи}\\
\textit{10.07.16}
\end{center}

\epigraph{\it Чем отличается ученик математического класса от ученика географического, экономического, политологического класса? Тем, что он больше размышляет над задачами? Да, и этим тоже. Но не только. Ещё он знает малую теорему Ферма...}{В. Сендеров, А. Спивак, ``Малая теорема Ферма'' (журнал ``Квант'', 2000, №1)}

\begin{problems}

\item Пусть $a$~--- некоторое число, которое не делится на простое число $p$.

\textbf{a)} Докажите, что в последовательности $0\cdot a$, $1\cdot a$, $2\cdot a$,\ldots,$(p-1)\cdot a$ все числа дают разные остатки по модулю $p$.

\textbf{b)} Докажите, что ~$(1\cdot a) \cdot (2\cdot a) \cdot \ldots\cdot ((p-1)\cdot a) \mathop{\equiv}\limits_p (p-1)!$.

\textbf{c)} (Малая теорема Ферма) Докажите, что $a^{p-1} \mathop{\equiv}\limits_p 1$.

 
\item Найдите остаток от деления $23^{1600}$ на $41$.
\item Докажите, что $300^{3000}-1$ делится на $1001$.
\item Докажите, что  $n^7-n$ делится на $42$. 
\item Докажите, что либо $n^{18}-1$, либо $n^{18}+1$ делится на $37$.
\item Докажите, что число $40^{81}+17^{160}$ является составным.
\item Пусть $p$~--- простое число.\\ \textbf{a)}Докажите, что для любых чисел $a$ и $b$ верно, что $(a+b)^p \mathop{\equiv}\limits_p a^p+b^p$.\\
\textbf{b)} Выведите из этой задачи малую теорему Ферма.
\item 	Математические хулиганы Гриша и Андрей катаются на лифте $17$-этажного дома. Они садятся в лифт на этаже с номером $n$,  и едут на этаж, номер которого равен остатку от деления  на $17$. После этого  они умножают на n номер этажа, на котором оказались, и едут на этаж с номером, равным остатку от деления на $17$ полученного произведения. На каком этаже может закончиться пятнадцатая  поездка?

\item Андрею и Грише надоело вспоминать, с какого этажа они начали путешествие, и теперь они просто возводят в квадрат номер этажа, на котором находятся, и едут на этаж с номером, равным остатку от деления результата на $17$. Какое максимальное количество поездок удастся им совершить таким способом?
\end{problems}

\begin{center}
\textbf{Задачи на подумать}
\end{center}
\begin{problems}
\item Отметим на бумаге произвольным образом $p-1$ точку. Каждой точке сопоставим какой-то ненулевой остаток при делении на $p$. Проведём из остатка $k$ стрелочку в остаток $ka$. \\ \textbf{а)} Убедитесь, что из каждой точки выходит одна стрелочка, и в кажду точку входит одна стрелочка.\\ 
\textbf{б)} Поймите, что тогда все точки разбиваются на циклические маршруты. \\ 
\textbf{в)} Докажите, что у всех циклических маршртутов одна и та же длина, и она делит $p-1$.\\ 
\textbf{г)} Выведите отсюда малую теорему Ферма.

\item Пусть $p$ и $q$ различные простые числа. Докажите, что $p^q+q^p \mathop{\equiv}\limits_{pq} p+q$.
\item Докажите, что для любого простого $p>5$ справедливо, что
\\ \textbf{a)} число $\underbrace{111\ldots11}_{p-1}$ делится на $p$;
\\ \textbf{b)} число $\underbrace{111\ldots11}_{p}$ не делится на $p$.
\item Найти все такие простые числа $p$, что число $5^{p^2}-1$ делится на $p$.  
\item Докажите, что для любого простого $p$ число $2^{2^p}-4$ делится на $2^p-1$. 
\item Может ли число $2^{1260} + 3^{1260} - 1$ быть точной десятой степенью?
%простите, что без $$


%Определение: Графом умножения на остаток n по модулю m называется ориентированный граф, вершинами которого служат остатки от деления на m, а из каждой вершины k единственное ребро ведет в вершину kn.

%\item а) Постройте графы умножения на 2, 3 и 5 по модулю 10, на 3 по модулю 9, на 1, 2, 3, 4, 5, 6 по модулю 7 .

%б) Докажите, что граф умножения на ненулевой остаток по простому модулю разлагается на циклы равной длины.

%в) Используя результат п. б), докажите малую теорему Ферма.

%\item 	Математические хулиганы Гриша и Вова катаются на лифте 17-этажного дома. Они садятся в лифт на этаже с номером n,  и едут на этаж, номер которого равен остатку от деления  на 17. После этого  они умножают на n номер этажа, на котором оказались, и едут на этаж с номером, равным остатку от деления на 17 полученного произведения. На каком этаже может закончиться пятнадцатая  поездка?

%\item Вове и Грише надоело вспоминать, с какого этажа они начали путешествие, и теперь они просто возводят в квадрат номер этажа, на котором находятся, и едут на этаж с номером, равным остатку от деления результата на 17. Какое максимальное количество поездок удастся им совершить таким способом?


\end{problems}

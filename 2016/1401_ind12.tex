\begin{center}
\textbf{\Large Индукция-2}\\
\textit{14.07.16}
\end{center}

\epigraph{\it Элемент тока длины $dl$ создаёт поле с магнитной индукцией $dB = k\frac{Idl}{r^2}$.}{Закон Био-Савара-Лапласа}

{\bf Задачи на разбор}
\begin{problems}

\item Плоскость поделена на области несколькими прямыми. Докажите, что эти области можно раскрасить в два цвета так, чтобы любые две соседние области были раскрашены в различные цвета.
\item У бородатого многоугольника во внешнюю сторону растет щетина. Его пересекает несколько прямых, на каждой из которых с одной из сторон тоже растет щетина. В результате многоугольник оказался разбитым на некоторое число частей. Докажите, что хотя бы одна из частей окажется бородатой снаружи (никакие три прямые не проходят через одну точку).
\item Докажите, что $3^{2n+2}+8n-9$ делится на 16.
\item Докажите, что любое число можно представить в виде суммы степеней двоек единственным образом.

\end{problems}
\resetproblem

{\bf Задачи для самостоятельного решения}
\begin{problems}
\item
Докажите равенство $$3 + 33 + 333 + \ldots + \underbrace{33 \ldots 33}_n = \dfrac{10^{n+1} - 9n - 10}{27}.$$
\item Числа вида $F_n = 2^{2^n} + 1$ называются числами Ферма. Докажите, что десятичная запись числа $F_n$ при $n \geqslant 2$ оканчивается цифрой 7.
\item Треугольник разбит на несколько частей несколькими прямыми. Докажите, что хотя бы одна из частей является треугольником.
\item Доказать, что монетами по 3 и 5 рублей можно выдать любую сумму больше 8 рублей?
\item Проведём в выпуклом многоугольнике некоторые диагонали так, что никакие две из них не пересекаются (из одной вершины могут выходить несколько диагоналей). Доказать, что найдутся по крайней мере две вершины многоугольника, из которых не проведено ни одной диагонали.
\item Докажите, что все числа 10017, 100117, 1001117, \ldots делятся на 53.
\item На столе стоят $2^n$ стаканов с водой. Разрешается взять любые два стакана и уравнять в них количества воды, перелив часть воды из одного стакана в другой. Докажите, что с помощью таких операций можно добиться того, чтобы во всех стаканах было поровну воды. 
\item Кусок бумаги разрешается рвать на 4 или на 6 кусков. Докажите, что по этим правилам его можно разорвать на любое число кусков, начиная с шести.
\item Докажите, что любое число можно представить в виде суммы нескольких различных чисел Фибоначчи. Единственно ли такое представление?
\item На кольцевой дороге стоят $n$ машин. При этом, суммарное количество топлива у машин  равно количеству топлива чтобы проехать один круг. Докажите, что существует машина, которая может проехать весь круг, забирая топливо у других машин.
\item Докажите, что $3^{2^n} - 1$ делится на $2^{n+2}$, но не делится на $2^{n+3}$.

\item Докажите, что $3^{2^n} - 1$ делится на $2^{n+2}$, но не делится на $2^{n+3}$.
\item Докажите,  что любое количество квадратов можно разрезать на несколько частей, из которых можно составить квадрат.
\item Даны два взаимнопростых числа $m$ и $n$, а также число 0. Имеется калькулятор, который умеет выполнять лишь одну операцию: вычислять среднее арифметическое двух чисел одной чётности. Докажите, что с помощью этого калькулятора можно получить все числа от $1$ до $n$.
\item  На окружности взяли $n$ точек и соединили их всевозможными отрезками. Оказалось, что никакие три не пересекаются в одной точке. На сколько частей они делят круг?
\end{problems}

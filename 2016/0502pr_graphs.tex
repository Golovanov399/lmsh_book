\begin{center}
\Large

\textbf{Графы-1. 05 Июля.}
\end{center}  

\large
\def\q#1.{{\smallskip\bf #1.}}

\q1. На банкете встретились 25 бизнесменов. После банкета каждый из
них пришел домой и послал по подарку каким-то семи из остальных.
Верно ли, что обязательно найдутся два бизнесмена, первый из которых
послал подарок второму, а второй первому~--- нет?


\q2. Чтобы отвести на завтрак, 100 детей построили парами. На обратном 
пути из столовой их снова построили парами, возможно, составленными по-другому. 
При каком наибольшем n наверняка можно выбрать $n$ 
детей, никакие два из них не были в одной паре?

\q3. В стране 12 городов, причем любые два из них соединены дорогой.
10 дорог закрыли на ремонт. Докажите, что из любого города можно
доехать до любого другого по этим дорогам.

\q4. В стране несколько городов, некоторые пары городов соединены
беспосадочными рейсами одной из $N$ авиакомпаний, причем из каждого
города есть ровно по одному рейсу каждой из авиакомпаний. Известно,
что из любого города можно долететь до любого другого (возможно, с
пересадками). Из-за финансового кризиса был закрыт $N-1$ рейс,  но
ни в одной из авиакомпаний не закрыли более одного рейса. Докажите,
что по-прежнему из любого города можно долететь до любого другого.

\q5. В стране 96 городов, из которых 24 -- областные, некоторые пары
городов соединены между собой дорогами (но не более чем одной),
причём любой путь по дорогам между двумя обычными городами, если он
есть, проходит хотя бы через один областной город. Какое наибольшее
количество дорог может быть в этой стране?

\q6. В стране 100 городов, из каждого города выходит хотя бы одна
дорога. Докажите, что можно закрыть несколько дорог так, чтобы
по-прежнему из каждого города выходило не менее одной дороги и при
этом по крайней мере из 67 городов выходило ровно по одной дороге.



\q7. На вечеринку пришло 2016 пар гостей (каждая пара состоит из
мальчика и девочки). Известно, что каждый гость кроме своего
партнера знаком еще хотя бы с одним гостем противоположного пола.
Докажите, что организаторы вечеринки могут раздать всем гостям шляпы
трех цветов так, что у каждого гостя будет хотя бы два знакомых в
шляпах разного цвета.

\q8. На планете 10000 городов, среди которых есть столицы
государств. Некоторые города связаны дорогами так, что любая дорога
соединяет ровно два города, и от любого города до любого другого
можно добраться по дорогам. При этом, чтобы попасть из одной столицы
в другую, нужно проехать не менее 200 дорог. Докажите, что на
планете меньше 100 столиц.

\q9. В кружок записались 15 мальчиков и 15 девочек, причем каждый из
мальчиков знаком хотя бы с тремя девочками. Докажите, что можно
выбрать шестерых мальчиков и двух девочек так, чтобы каждый из
выбранных мальчиков был знаком хотя бы с одной из выбранных девочек.

\begin{center}
\Large
\textbf{Подсчет числа способов. Треугольник Паскаля. 07 июля.}
\end{center}  

\def\q#1.{{\smallskip\bf #1.}}
\q0. a) Имеется $m$ белых и $n$ черных шаров, причем  $m > n$.
Сколькими способами можно все шары разложить в ряд так, чтобы
никакие два черных шара не лежали рядом? \\
b) Cколькими способами можно рассадить m разноцветных попугаев по n клеткам так, чтобы в каждой сидели один или два попугая?\\
c) Сколькими способами можно выбрать $k$ различных подмножеств $n$-элементного множества?\\
d) Сколькими способами можно выбрать $k$ непересекающихся множеств из $n$ элементного множества?

\medskip

{\bf Напоминание.} Число способов выбрать из~$n$ предметов
произвольные~$k$ равно~$\frac{n!}{(n-k)!k!}$ и обозначается~$C_n^k$.

\q1. Докажите следующие равенства: \\
 a) $kC_n^k=nC_{n-1}^{k-1}$\\
 b) $C_n^k=C_{n-1}^k + C_{n-1}^{k-1}$;\\
 c) $C_r^mC_m^k=C_r^kC_{r-k}^{m-k}$

\q2. ({\bf Бином Ньютона}) Докажите формулу
$$(x+y)^n=\sum_{k=0}^nC_n^kx^ky^{n-k}.$$
(коэффициенты $C_n^k$ называются {\it биномиальными}.)

\q3. Найдите следующие суммы
\begin{equation*}
a) \sum_{k=0}^n C_n^k; \quad b) \sum_{k=0}^n 2^kC_n^k; \quad c)
\sum_{k=0}^n (-1)^kC_n^k; \quad d) \sum_{k=0}^n kC_n^k \quad
\end{equation*}
%над последним пунктом засядут


\q4. Дана клетчатая доска размера $n \times m$, в левом нижнем углу
которой стоит фишка. За один ход можно переместить фишку на одну
клетку вверх или вправо. Сколькими способами можно добраться до
правого верхнего угла доски?



\medskip

{\bf Опр.} {\it Треугольник Паскаля}~--- бесконечная таблица
биномиальных коэффициентов, имеющая треугольную форму. В этом
треугольнике на вершине и по бокам стоят единицы. Каждое число равно
сумме двух расположенных над ним чисел.

\medskip

\q5. Докажите, что каждое число $a$ в треугольнике Паскаля,
уменьшенное на~1, равно сумме всех чисел, заполняющих
параллелограмм, ограниченный теми правой и левой диагоналями, на
пересечении которых стоит число $a$ (сами эти диагонали в
рассматриваемый параллелограмм не включаются).

\q6. Дан треугольник Паскаля из $n$ строк. Найдите сумму всех его
элементов.

\q7. {\it Диагональю} треугольника Паскаля называется множество
чисел, расположенных на прямой, параллельной его стороне. Найдите
сумму чисел, стоящих на $k$-ой диагонали треугольника из $n$ строк.

\q8. Докажите равенство:
$1^2+2^2+\ldots+k^2=C_{k+1}^2+2(C_k^2+C_{k-1}^2+\ldots+C_2^2)$.

\q9. Докажите, что не существует таких натуральных чисел $x$, $y$,
$z$, $k$, что $x^k + y^k = z^k$  при условии  $x < k,  y < k$.

\q10. В левом столбце и нижней строке таблицы $11 \times 11$ лежат
фишки~--- всего $2^{15}$ фишек. За один ход разрешается выбрать две
граничащие по углу клетки, снять с них по фишке и добавить фишку в
клетку, имеющую общую сторону с выбранными. Можно ли добиться того,
чтобы хотя бы одна фишка попала в правый верхний угол?

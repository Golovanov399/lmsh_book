\begin{center}
\textbf{\Large Делимость}\\
%\textit{Профи}\\
\textit{04.07.16}
\end{center}

\epigraph{\it Как зарплату делить будем? Поровну, по-честному, по-братски или по справедливости?}{@shhdup}

\begin{problems}
\item Пусть $a$~--- чётное число, не кратное $4$. Докажите, что разность $a^2 - 4$ делится на $32$.
\item Известно, что $5x+8y-1$ делится на $13$.\\ \textbf{a)} Докажите, что $5x+60y-1$ делится на $13$.\\ \textbf{b)} Найдите остаток от деления $18x-31y$ на $13$. \\ \textbf{c)} Найдите остаток от деления $x-y$ на $13$.
\item Вася написал на доске два числа, перемножил их и получил четырёхзначное число. После этого он заменил буквы на числа, причём разным числам соответствуют разные буквы. В итоге получилось $AB \cdot CD = EEFF$. Докажите, что Вася ошибся.
\item На очень большой и длинной доске записано число $11^{2016}$. Потом вместо этого числа записали сумму его цифр. Затем снова вместо полученного числа записали сумму его цифр. Этот процесс продолжается до тех пор, пока не останется однозначное число. Найдите это число. 
\item На доске записаны два числа: единица и двойка. Каждую минуту Маша умножает два самых больших числа, написанных на доске, прибавляет к ним $1$ и записывает полученное число на доску. Докажите, что Маша никогда не запишет число, делящееся на $4$.
\item Пусть $k>2$~--- нечётное натуральное число. Докажите, что для любого натурального $n$ число $1^{k^n}+2^{k^n}+...+(k-1)^{k^n}$ делится на $k$.
\item Может ли $5^n-1$ делиться на $4^n-1$ при натуральном $n$? 
\item В ряд записана последовательность чисел $a_n$, причём оказалось, что для любого числа, начиная с третьего, справедлива формула $a_n=a_{n-2}+2a_{n-1}$. Оказалось, что первые два члена~--- простые числа. Какое максимальное количество подряд идущих простых чисел может еще встретиться после них?
%\item Можно ли числа $1, 2, 3, . . . , 20$ так расставить в вершинах и серединах ребер куба так, чтобы каждое число, стоящее в середине ребра, равнялось полусумме чисел на концах этого ребра? (не совсем делимость, халява)
% \item Назовем число $n$ \textit{удобным}, если $n^2+1$ делится на $1000001$. Докажите, что среди чисел $1, 2, ..., 1000000$ четное число удобных.
% \item Назовем число совершенно простым, если оно простое и если при любой перестановке его цифр снова получается простое число. Докажите, что в записи абсолютно простого числа не может содержаться более трех различных цифр.
% \item Из трех различных цифр составили шесть различных двузначных чисел (каждая цифра входит в запись по одному разу). Докажите, что если сложить какие-то три из этих чисел и вычесть сумму трех оставшихся, то полученный ответ будет делиться на $18$. 

\end{problems}

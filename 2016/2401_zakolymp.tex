\begin{center}
\textbf{\Large Заключительная олимпиада}\\
%\textit{Профи}\\
\textit{24.07.16}
\end{center}


\begin{problems}

\item Владимир Алексеевич за обедом съел суп в два раза быстрее, чем Леонид Андреевич, а котлету --- в два раза медленнее. При этом он закончил обед раньше. Выясните, пожалуйста, что происходило дольше: Владимир Алексеевич ел суп или Леонид Андреевич ел котлету.

\item Числа от 1 до 100 выписаны в ряд в некотором порядке. Сорок
из них покрашены в белый цвет, остальные~--- в черный. 
Докажите, пожалуйста, что найдутся либо два соседних одноцветных числа, сумма
которых нечетна, либо два соседних разноцветных числа, сумма
которых четна.

\item Решите, пожалуйста, в натуральных числах уравнение
$$
abc+ab+c=a^3.
$$

\item На сторонах $AB$ и $AC$  равностороннего треугольника $ABC$  выбраны точки $P$ и $Q$  соответственно 
таким образом, что и $AP = CQ$. Точка $M$~---  середина $PQ$. Докажите, пожалуйста, что $2AM=CP$.  

\item Рассмотрим фигуру, образованную всеми клетками диагонали квадрата
$20\times 20$, идущей из правого нижнего угла в левый верхний, а
также всеми клетками, лежащими выше этой диагонали. 
Найдите, пожалуйста, сколькими
способами эту фигуру можно разбить на клетчатые прямоугольники с
попарно различными площадями?

\item Решите, пожалуйста, уравнение в натуральных числах
$$
2^a+3^b+1=6^c.
$$

\item В чемпионате по волейболу участвовало несколько команд, каждые две из которых сыграли друг с другом ровно один раз. Оказалось, что для каждых двух команд есть ровно 59 команд, у которых они обе выиграли. Докажите, пожалуйста, что количество команд равно 239.  Напомним, что ничьих в волейболе не бывает.
	
\item Внутри острого угла $AOB$ дана точка $M$, а на разных сторонах этого угла выбраны точки $P$ и
$Q$ таким образом, что $OP +OQ = OM$ и сумма отрезков $MP$ и $MQ$  наименьшая для всех возможных
таких пар точек $P$ и $Q$. Докажите, пожалуйста, что $\angle OPM = \angle OQM$.

\end{problems}

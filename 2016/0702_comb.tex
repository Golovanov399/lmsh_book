\begin{center}
\textbf{\Large Классическая комбинаторика}\\
%\textit{Профи}\\
\textit{07.07.16}
\end{center}

\epigraph{\it Это знать надо! Это классика!}{Товарищ капитан}

\textit{Материалы для работы по группам на второй паре. Цель работы~--- доказать следующие равенства комбинаторно. У разных групп разные наборы неравенств.}

\begin{problems}
\item $kC_n^k=nC_{n-1}^{k-1}$

\item $C_n^0\cdot 2^0+C_n^1\cdot 2^1+\cdots+C_n^n\cdot 2^n=3^n$

\item $C_n^0 C_n^m + C_n^1 C_{n-1}^{m-1} + \cdots + C_n^m C_{n-m}^{n-m}=2^m C_n^m$

\item $C_k^k+C_{k+1}^k+\cdots+C_n^k=C_{n+1}^{k+1}$

\item $C^1_n + 6C^2_n + 6C^3_n = n^3$
\end{problems}
\resetproblem
\vspace{5pt}
\hrule
\begin{problems}
\item $C_n^{k-1}+C_n^{k}=C_{n+1}^k$

\item $C_p^0 C_q^k + C_p^1 C_q^{k-1}+\cdots +C_p^{k-1} C_q^1 ++C_p^k C_q^0=C_{p+q}^k$

\item $0\cdot C_n^0 + 1\cdot C_n^1 + 2\cdot C_n^2+\cdots+n\cdot C_n^n=n\cdot 2^{n-1}$

\item $C_{n-1}^0+C_n^1+C_{n+1}^2+\cdots+C_{n+m-1}^m=C_{n+m}^m$

\item $1 + 14C^1_n + 36C^2_n + 24C^3_n = (n + 1)^4 - n^4$
\end{problems}
\resetproblem
\vspace{5pt}
\hrule
\begin{problems}
\item $C_r^m C_m^k=C_r^k C_{r-k}^{m-k}$

\item $C_n^0+C_n^2+C_n^4+\dots=C_n^1+C_n^3+C_n^5+\dots=C_{n-1}^0+C_{n-1}^1+\cdots+C_{n-1}^{n-1}$

\item $(C_n^0)^2 + (C_n^{1})^2+\cdots+(C_n^n)^2=C_{2n}^n$

\item $C_{n-1}^0+C_n^1+C_{n+1}^2+\cdots+C_{n+m-1}^m=C_{n+m}^m$

\item $C^1_n + 14C^2_n + 36C^3_n + 24C^4_n = n^4$
\end{problems}
\resetproblem
\vspace{5pt}
\hrule

\begin{problems}
\item Сколькими способами могут выпасть три различные игральные кости?

a. Во скольких случаях хотя бы на одной кости будет 6 очков?

b. Ровно на одной будет 6 очков?

c. Ровно на одной 6, а ровно на одной другой 3 очка?

\item Имеется $n$ абонентов телефонной сети. Сколькими способами можно одновременно соединить 3 пары абонентов?

\item Сколькими способами можно выстроить в шеренгу 5 футболистов, 5 бегунов и 5 боксёров так, чтобы никакие два боксёра не стояли рядом (отличить двух спортсменов, занимающихся одним видом спорта, невозможно).

\item В выпуклом $n$-угольнике никакие три диагонали не пересекаются в одной
точке. Найдите количество точек пересечения диагоналей.

\item Пусть $p>2$~--- простое число. Сколько существует способов раскрасить вершины правильного $p$-угольника в $a$ цветов? (раскраски, которые можно совместить поворотом, считаются одинаковыми.) 

\item Пусть $p>2$~--- простое число. Сколькими способами можно провести через вершины правильного $p$-угольника замкнутую ориентированную $p$-звенную ломаную? (ломаные, которые можно совместить поворотом, считаются одинаковыми.) 

\end{problems}

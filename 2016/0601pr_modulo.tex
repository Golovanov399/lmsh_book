\begin{center}
\Large
\textbf{Сравнения. 06 июля}
\end{center}

{\bf Определение.} Говорят, что целые числа $a,b$ сравнимы по модулю $n$ ($n$ -- натуральное число), если $a-b$ делится на $n$. Обозначается как $a\equiv b (mod \quad n)$

{\bf Свойства. } Пусть $a, b, c, d, e \in \mathbb{Z}, m \in \mathbb{N}$. Если $a \equiv b$ (mod $n$), $c \equiv d$(mod $n$), то:

1) $a+c\equiv b+d$ (mod $n$), $ae\equiv be$ (mod $n$);

2) $ac\equiv bd$ (mod $n$), $a^{m}\equiv b^{m}$ (mod $n$).

3) Если $k$ и $m$ взаимно простые и $ka \equiv kb \quad (mod \quad m)$, то $a \equiv b \quad (mod \quad n)$. (Для доказательства этого свойства пока что ссылаемся на основную теорему арифметики.)

4) Целые числа $a,b$ сравнимы по модулю $n$ тогда и только тогда, когда они имеют одинаковый остаток при делении на $n$ (это не очевидный факт).


\bigskip

{\it О работе со сравнениями, как с уравнениями}

\bigskip


\begin{enumerate}


\item Решите сравнения:\\
а) $5x\equiv 2 \quad  (mod \quad  3)$;\\
б) $3x\equiv 2 \quad (mod \quad  11)$;\\
в) $6x\equiv 1 \quad (mod \quad  13)$;



\item 
а) Пусть $k \in \mathbb{N}$, $n \in \mathbb{N}, 1 \leq n \leq k-1$, $n$ взаимно просто с $k$. Докажите, что существует единственное $m\in \mathbb{N}, 1\leq m \leq k-1$, такое, что $nm\equiv 1$ (mod $k$).

б) Пусть $p\in \mathbb{P},x\in \mathbb{Z}$. Докажите, что если  $x^{2}\equiv 1$ (mod $p$), то либо $x\equiv 1$ (mod $p$), либо $x\equiv -1$ (mod $p$). 

в) ({\bf Теорема Вильсона}) Докажите, что если $p\in \mathbb{P}$, то $(p-1)!\equiv -1$ (mod $p$).


\item Даны натуральные числа $a$, $b$ и $c$ такие, что $ab+9b+81$ и $bc+9c+81$ делятся на $101$. Докажите, что тогда и $ca+9a+81$ тоже делится на 101. 

\bigskip

{\it Перемножаем, складываем.}

\bigskip



\item Натуральные числа $a$ и $b$ таковы, что $ab+1$ делится на $b+2$. Докажите, что $2a > b$.


\item Пусть $a, b, c, d$ и $n$ — натуральные числа, причем $a+b$  и $c+d$ делятся на $n$. Докажите, что $ac-bd$ делится на $n$.


\item а) Докажите, что $a^{n}-b^{n}$ делится на $a-b$.\\
б) Докажите, что при нечётном $n$ $a^n+b^n$ делится на $a+b$.

%$\textit{Якобы практические}$

\item Докажите, что $(2^n-1)^n-3$ делится на $2^n-3$.

\item Вася выписал в тетрадку числа вида $100\ldots 01$ (иными словами, числа вида $10^k+1$), меньшие $10^{2015}$. Докажите, что простых из них не более 1\% от общего числа.




\item Натуральные числа $a$ и $b$ таковы, что $a^{16}+b^{16}$ и $a^{125}+b^{125}$ делятся на 101. Докажите, что $a^{2016}+b^{2016}$ делится на 101.

\item Число $1\underbrace{33...33}_k$ -- простое. Докажите, что $k$ -- нечетное. 

\bigskip

{\it Добавочка.}

\bigskip

\item  Дано четное число $a$. Докажите, что существует бесконечно много нечетных натуральных чисел $n$ таких, что $a^{n} + n$ — составное число.


\item В ряду чисел $1, 501, 751, 876, 438, . . .$ каждое число, кроме первого, равно половине предыдущего, если предыдущее четно, и половине предыдущего числа, увеличенного на 1001, в противном случае. Верно ли, что в этом ряду встретятся все натуральные числа от $1$ до $1000$?


\end{enumerate}
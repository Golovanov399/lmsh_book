\begin{center}
\Large
\textbf{Теория чисел. Малая теорема Ферма. 10 июля.}
\end{center}

{\large
\q1т. а)~Пусть $a,b \in \mathbb Z$ и~$p \in \mathbb P$. Докажите,
что~$(a+b)^p \equiv a^p+b^p \pmod{p}$.

б)~Выведите отсюда по индукции { \it малую теорему Ферма}: если $p\in \mathbb{P}, a\in \mathbb{Z}$ и $a$ не делится на $p$, то $a^{p-1}\equiv 1$ (mod $p$).




\q2т.  Пусть $a \in \mathbb{Z}, p \in \mathbb{P}$ и $a$ не делится на $p$. Рассмотрим следующий ориентированный граф: вершины графа — числа от 1 до $p-1$; из числа $x$ ведет ориентированное ребро в число $y$, если $ax \equiv y$ (mod $p$). 

а) Докажите, что этот граф является объединением нескольких циклов одинаковой длины.

б) Выведите из этого { малую теорему Ферма}.

\q3т. Возьмём натуральное число $a$, которое не делится на простое число $p$. 


а) Докажите, что среди остатков $a, 2a, 3a, \ldots (p-1)a$ есть все ненулевые остатки по модулю~$p$. 


б) Перемножив всё это, выведите малую теорему Ферма. 

\q4. Пусть $p$ и $q$ --- различные простые числа. Докажите, что $p^q+q^p \equiv p+q \pmod  {pq}$.


\q5. Какой остаток даёт число $42^{42^{42}}$ при делении на 2017?

\q6.  Дано простое число $p$. Докажите, что $2^{2^p}-4 $ делится на $2^p-1$.

\q7. Докажите, что если $a^{15}-1$ делится на 29, то и $a-1$ делится на 29.

\q8. Какие остатки может давать число $a^{50}$ при делении на 101?}

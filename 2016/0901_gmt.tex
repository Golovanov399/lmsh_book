\renewcommand{\baselinestretch}{0.8}
\parskip=0.8\parskip

\begin{center}
\textbf{\Large ГМТ}\\
%\textit{Профи}\\
\textit{09.07.16}
\end{center}

\epigraph{\it В общем случае, геометрическое место точек формулируется параметрическим предикатом, аргументом которого является точка данного линейного пространства.}{Википедия}

\textbf{Обозначение.} {\itГеометрическим местом точек (ГМТ)} называется фигура, состоящая из все точек плоскости, которые удовлетворяют какому-либо определенному условию.

%примеры: окружность, круг



\begin{problems}

\item   a) Найдите ГМТ, равноудаленных от точек $A$ и $B$;\\
b) Найдите ГМТ, равноудаленных от двух заданных прямых;\\
c) Найдите ГМТ, равноудаленных от сторон заданного угла;\\
d) Дан отрезок $AB$. Найдите ГМТ таких, что $AM<BM$.

%\item   Дан четырехугольник $ABCD$. Где находится такая точка $O$, что $AO=CO$ и $BO=DO$? Сколько может быть таких точек?

\item  Дан четырехугольник $ABCD$. Оказалось, что существуют две такие точки $O$, что $AO=DO$ и $BO=CO$. Докажите, что $AD$ и $BC$ параллельны.

\item  Дан отрезок $AB$.  Найдите ГМТ $M$ таких, что $AM$ --- наименьшая сторона треугольника $ABM$ (ответ дайте в виде заштрихованной области).

\item   Даны точки $A$ и $B$. Найдите ГМТ $M$ таких, что угол $ \angle BAM< \angle AMB < \angle ABM$.

%\item  Дан треугольник $ABC$. Некоторая точка $M$ такова, что $AM=1$, $BM=2$, $ CM=3$. Единственна ли такая точка $M$?

\item   На прямой выбран отрезок $AB$. Из точки $D$ на этой прямой, но вне отрезка, восстановлен перпендикуляр $CD$ такой, что $AB=CD$. Сколько существует точек $M$ таких, что треугольники $ABM$ и $CDM$ равны?

%\item  Точка $O$ --- середина отрезка $MK$. Известно, что $AM<BM$,  $AK<BK$. Докажите, что $AO<BO$.

\item   Диагонали четырехугольника равны. Известно, что серединный перпендикуляр к одной его стороне пересекает противоположную сторону. Докажите, что это верно и для противоположной стороны.

\item   Дан шестиугольник, никакие стороны которого не параллельны. Стороны покрашены в черный и белый цвет по очереди. Сколько существует точек, которые равноудалены от всех черных сторон?

%\item  В четырехугольнике $ABCD$ внешний угол при вершине $A$ равен углу $BCD$, $AD=CD$. Докажите, что $BD$ --- биссектриса угла $ABC$.

%\item   Дан квадрат $ABCD$. Найдите ГМТ, сумма расстояний от которых до прямых $AB$ и $CD$ равна сумме расстояний до прямых $AD$ и $BC$.

\item   Пусть $O$ --- центр правильного треугольника $ABC$. Найдите ГМТ $M$, удовлетворяющих следующему условию: любая прямая, проведенная через точку $M$, пересекает либо отрезок $AB$, либо отрезок $CO$.

%\item   Даны прямая $l$ и точки $A$ и $B$ по одну сторону от нее. Найдите ГМТ $M$ таких, что прямая $AM$ пересекает прямую $l$ левее, чем прямая $BM$.

% с выходом на среднюю линию
\item На сторонах $AB$ и $BC$ треугольника $ABC$ берутся точки $D$ и $E$ (по одной на каждой). Найдите геометрическое место середин отрезков $DE$.

% знаю только сложное решение с окружностями, а должно быть и доброе
\item Найдите геометрическое место точек $M$, лежащих внутри ромба $ABCD$ и обладающих тем свойством, что  $ \angle AMD +  \angle BMC = 180^\circ$.


\end{problems}
\renewcommand{\baselinestretch}{1}
\parskip=1.25\parskip

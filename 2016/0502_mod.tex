\begin{center}
\textbf{\Large Сравнения}\\
%\textit{Профи}\\
\textit{05.07.16}
\end{center}

\epigraph{\textit{Посмотрите на своего мужчину, а теперь на меня. И снова на своего мужчину, и снова на меня.}}{Реклама Old Spice}

\textbf{Определение.} Если два числа дают одинаковые остатки при делении на число $n,$ то говорят, что они сравнимы по модулю $m.$ Записывают это так: ~$a\mathop{\equiv}\limits_m b$ или $a\equiv b\,(mod\,m).$\\
\textbf{Упражнение 1.} Числа $a$ и $b$ сравнимы по модулю $n$ тогда и только тогда, когда число $a-b$ сравнимо с $0$ по модулю $n$.\\
\textbf{Свойства сравнений:}\\
a) если $a \mathop{\equiv}\limits_m b ,$ $b \mathop{\equiv}\limits_m c,$ то $a \mathop{\equiv}\limits_m c  ;$\quad  b) $a \mathop{\equiv}\limits_m a + km  ,$ где $k$~--- целое число;

c) если $a \mathop{\equiv}\limits_m b  ,$ то $a + c  \mathop{\equiv}\limits_m b + c  ;$\quad d) если $a \mathop{\equiv}\limits_m b$   и $c \mathop{\equiv}\limits_m d  ,$ то $a + c \mathop{\equiv}\limits_m b + d  ;$

e) если $a \mathop{\equiv}\limits_m b,$   то $ac \mathop{\equiv}\limits_m bc  ;$\quad f) если $a \mathop{\equiv}\limits_m b$ и $c \mathop{\equiv}\limits_m d  ,$ то $ac \mathop{\equiv}\limits_m bd  ;$



\begin{problems}

\item Доказать упражнение и все свойства сравнений. 
\item Докажите, что:

\textbf{a)} если $a  \mathop{\equiv}\limits_m b,$   то $a^k  \mathop{\equiv}\limits_m b^k,$  где $k$~--- натуральное число;

\textbf{b)} привести пример, когда $ac \mathop{\equiv}\limits_m bc  ,$ но не выполняется $a \mathop{\equiv}\limits_m b  .$

\textbf{c)} сформулируйте, когда можно сокращать на одно и то же число обе части сравнения, и докажите это свойство.
\item Найдите остаток от деления:\\
\textbf{a)} $2015 \cdot 2016 \cdot 2017 \cdot 2018 \cdot 2019$ на $11$.\\
\textbf{b)} $1001 \cdot 1002 \cdot 1003+2001 \cdot 2002 \cdot 2003\cdot 2004$ на $1000$.\\
\textbf{c)} $2015 \cdot 2014 \cdot 2013+2017 \cdot 2018 \cdot 2019$ на $2016$.
\item Найдите остаток от деления:\\
\textbf{a)} $9^{2016}+13^{2016}$ на $11$.\\
\textbf{b)} $9^{2015}+13^{2015}$ на $11$.
\item Докажите, что \textbf{a)} $2^{2016} \mathop{\equiv}\limits_5 3^{2016}$; \textbf{b)} $2^{2016} \mathop{\equiv}\limits_{13} 3^{2016}$;  \textbf{c)} найдите еще хотя бы одно простое число $p$, для которого $2^{2016} \mathop{\equiv}\limits_{p} 3^{2016}$.
\item  Пусть $A$~--- произведение всех нечётных чисел от $1$ до $2017$, а $B$~--- произведение всех чётных чисел от $2$ до $2018$. Докажите, что $A + B$ делится на $2019$.

\end{problems}

\begin{center}
\textbf{На подумать}\\
\end{center}

\begin{problems}
\item Докажите, что число $(5^n-1)^n-6$ делится на $5^n-6$. 

\item Радиолампа имеет $1001$ контакт, расположенных по кругу и включаемых в штепсель, имеющий $1001$ отверстие. Можно ли так занумеровать контакты лампы и отверстия штепселя, чтобы при любом включении лампы хотя бы один контакт попал на свое место (т.е. в отверстие с тем же номером)?  

\item Первоклассник Петя знает только цифры $1$ и $2$. Докажите, что он может написать число, делящееся на $123456789$.

\item Число $1\underbrace{33...33}_k$~--- простое. Докажите, что $k$~--- нечетное. 
\end{problems}

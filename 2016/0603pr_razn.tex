\begin{center}
\Large
\textbf{Разнобой. 06 июля}
\end{center}

\begin{enumerate}

\item Три простых числа таковы, что квадрат суммы любых двух даёт остаток единицу при делении на третье. Докажите, что какие-то два из чисел равны.


\item $ABCD$~--- выпуклый четырехугольник, в котором $\angle CAD + \angle BCA = 180^{\circ}$ и $AB = BC + AD$. Докажите, что $\angle BAC + \angle ACD = \angle CDA$. 


\item На день рождения к Арсению пришли 12 друзей и расселись за круглым 
столом в ожидании вкусного угощения. После этого перед ними разложили в 
каком-то  порядке карточки с их именами (все имена различны). 
Докажите, что  Арсений может повернуть стол так, чтобы хотя бы двум гостям 
достались карточки с их собственными именами.  

\item Решите в целых числах уравнение $6x+7y=xy-1$.

\item Рассмотрим все треугольники с вершинами в вершинах выпуклого
2016-угольника. Докажите, что любая точка, не лежащая на сторонах таких
треугольников, покрыта четным числом из них.

\item  На бесконечном клетчатом листе двое играют в крестики-нолики по следующим
правилам: первый игрок своим ходом может поставить два крестика, а второй
своим ходом один нолик. Первый игрок выигрывает, если поставит~100 крестиков
подряд. Может ли второй игрок ему помешать?
 

\item Для натурального $n$ введем обозначение 
$$\alpha(n)={(n+1)(n+2)\dots(n+20)\over \hbox{НОК}(n+1, n+2, \dots, n+20)}.$$ 
Докажите, что существуют различные $m, n > 1000$, для которых $\alpha(m)=\alpha(n)$.

\end{enumerate}
\begin{center}
\textbf{\Large Изоморфизм комбинаторных задач}\\
%\textit{Профи}\\
\textit{15.07.16}
\end{center}

\epigraph{\it Если $666$~--- это зло, то $25.807$~--- корень зла.}{Народная мудрость}

Не решая задач, разбейте их на группы <<одинаковых задач>>. Объясните, как установить соответствие между задачами одной группы.

\begin{problems}

\item Зуб может быть здоровым, больным, отсутствующим, растущим, искусственным и сломанным. У шушканчиков 6 рядов по 6 зубов. К черепашке Посе пришли в гости несколько шушканчиков с различными наборами зубов. Каково могло быть максимальное число гостей-шушканчиков? % 6^36

\item Сколькими способами можно поставить на доску $6\times6$ шесть ладей? % C

\item Сколько существует способов расставить 36 человек в шеренгу? % !

\item Сколькими способами можно на доске $36\times36$  расставить 36 ладей, не бьющих друг друга? % !

\item Есть 36 человек. Сколькими способами можно разбить их на 6 команд по 6 человек, если команды будут участвовать в разных соревнованиях? % multiC

\item Пося приехала в страну, где всего 64 города: каждый из них называется словом из 6 букв, состоящим только из букв <<Х>> и <<Е>>. Дорогой соединены ровно те пары городов, названия которых отличаются ровно в одной букве. Пося вышла из города ХХХХХХ и 36 раз прошла по дороге в другой город. Сколькими способами она могла это сделать? % 6^36

\item В магазине продаются чашки 6 видов и блюдца 6 видов. Сколькими способами можно выбрать 6 различных наборов из чашки и блюдца? % C

\item Имеется 30 ёжиков и 6 дикобразов. Сколько существует способов отправить по одному зверьку в 36 зоопарков? % C

\item Сколькими способами можно расставить на доске $6\times6$ числа от 1 до 36? % !

\item У Поси есть 6 револьверов, каждый из которых содержит по 6 пуль. Пули в первом револьвере отравляют, во втором~--- разрывают, в третьем~--- пронзают насквозь, в четвёртом~--- поджигают, в пятом~--- отчисляют из лагеря, а в шестом~--- расщепляют на молекулы. Пося хочет расстрелять 36 черепах по очереди: сначала первую, затем вторую, и так далее. Каждый выстрел можно делать из любого револьвера, в котором на этот момент есть хотя бы одна пуля. Сколькими способами Пося может это сделать? % multiC

\item В алфавите ЛМШ есть 6 букв: <<л>>, <<м>>, <<ш>> и <<ф>>, <<x>>, <<б>>. Сколько различных слов длины 36 есть в ЛМШ (любая последовательность букв в ЛМШ считается осмысленной)? % 6^36

\item Сколько существует способов раскраски доски $6\times6$ в 6 цветов? % 6^36

\item Есть 6 видов конфет, по мешку каждого вида. Сколько существует способов угостить ими 6 девочек так, чтобы ни одной не попалось более пяти одинаковых конфет? % 6^36

\item Есть 36-угольник, сколько существует шестиугольников с вершинами в вершинах 36-угольника? % C

\item Есть 36 разных конфет. Сколькими способами можно раздать их 36 девочкам по одной? % !

\item В стране черепашки Поси 117649 городов: Вишкиль-000000, Вишкиль-000001, и так далее до Вишкиля-666666 (в записи каждого используются только цифры от 0 до 6). Дорогой соединены ровно те пары городов, которые отличаются ровно в одной цифре на 1. Пося вышла из города Вишкиль-000000 и, двигаясь только по дорогам, посетила некоторые города в порядке возрастания номеров и остановилась в Вишкиле-666666. Сколькими способами она могла это сделать? % multiC

% \item (переделать)
% Черепашка Пося ходит по городу, в котором к каждому перекрёстку ведёт одна дорога, а отходит по 6. Пося собирается делать выбор на 6 перекрёстках. Сколько есть различных мест, куда она может попасть?
% \item (переделать)
% У Поси есть 6 волшебных игральных кубиков. Если такой кубик подкинуть, то результат выпадения повторится не раньше чем через сутки (если кубик не может выпасть ни на одну из граней, он испаряется). Пося достаёт кубики из мешочка, подкидывает, запоминает результат и возвращает в мешочек. Сколькими способами она может это сделать, пока все кубики не испарятся?

\end{problems}
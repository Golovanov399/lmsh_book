\begin{center}
\textbf{\Large Четвёртый <<признак равенства>> треугольников}\\
%\textit{Профи}\\
\textit{11.07.16}
\end{center}

\epigraph{\it Дело в том, что пока ты боишься высоты~--- она сильнее тебя.}{О. Тищенков, ``Кот''}

\begin{problems}

\item В треугольнике $ABC$  $\angle BAC = 15^{\circ}$. Можно ли однозначно найти сторону $AC$, если\\
а) $AB = 3$, $BC = 4$;\\
б) $AB = 4$, $BC = 3$?

\item В треугольниках $ABC$ и $A'B'C'$  $AB=A'B'$ и $\angle BAC+\angle B'A'C'=180^{\circ}$. Докажите, что равенство сторон $BC$ и $B'C'$ равносильно равенству углов $\angle BCA$ и $\angle B'C'A'$.

\item В выпуклом четырехугольнике $ABCD$: $AD = BC$; $\angle ABD + \angle CDB = 180^{\circ}$. Докажите, что $\angle BAD = \angle BCD$.

\item Пусть $K$~--- середина стороны $BC$ треугольника $ABC$. На лучах $AB$ и $AC$ взяты точки $X$ и $Y$ соответственно таким образом, что $AX = AY$ и точка $K$ лежит на отрезке $XY$. Докажите, что $BX = CY$.

\item  В четырехугольнике $ABCD$ внешний угол при вершине $A$ равен углу $BCD$, $AD=CD$. Докажите, что $BD$~--- биссектриса угла $ABC$.

\item В выпуклом четырёхугольнике $ABCD$, в котором $AB = CD$, на сторонах $AB$ и $CD$ выбраны точки $K$ и $M$ соответственно. Оказалось, что $AM = KC, BM = KD$. Докажите, что угол между прямыми $AB$ и $KM$ равен углу между прямыми $KM$ и $CD$.

\item В неравнобедренном треугольнике $ABC$ биссектрисы $AA_1$ и $BB_1$ пересекаются в точке $I$. Найдите $\angle C$, 
если $A_1I=B_1I$.

\item В выпуклом четырёхугольнике $ABCD$ $\angle A=\angle C$. На продолжении $BA$ за точкой $A$ выбрали точку $E$ такую, что $BC=AE$. Оказалось, что $\angle BDC=\angle ADE$ и $\angle ABD= \angle CAD$. Докажите, что $AC \perp BD$.

\item На стороне $BC$ равностороннего треугольника $ABC$ взята точка $M$, а на продолжении стороны $AC$ за точку 
$C$ --- точка $N$, причем $AM=MN$. Докажите, что $BM=CN$.

\item Внутри квадрата $ABCD$ выбрана точка $P$, а на его сторонах $BC$ и $CD$~--- точки $K$ и $L$ соответственно таким образом, что $PK = PL$. Точка $Q$ отрезка $BK$ такова, что $BP < PQ = DL$, а углы $BKP$ и $DPL$ равны. Докажите, что $PQ$ перпендикулярно $DP$.

\item На сторонах $BC$ и $AB$ треугольника $ABC$ выбраны точки $A'$ и $C'$ соответственно. Прямые $AA'$ и $CC'$ пересекаются в точке $K$. Оказалось, что $AC'=CA'$ и $\angle ABC=\angle A'KC$. На отрезке $KC'$ выбрана точка $P$ такая, что $2PK=A'K+KC'$. Докажите, что $\angle APC=90^{\circ}$.

%\item (было в Мск, гробина для 7классников) Обязательно ли треугольник равнобедренный, если точка пересечения биссектрис одинаково удалена от середин двух сторон?
\end{problems}

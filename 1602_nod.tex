\begin{center}
\textbf{\Large   Линейное представление НОД }\\
%\textit{Профи}\\
\textit{16.07.16}
\end{center}

\epigraph{\it Не всё в жизни смешно.}{Р. Л. Асприн}
 
\begin{problems}
\item Натуральные числа $a$ и $b$ взаимно просты. По окружности длины $a$ катится колесо
длины $b$, в обод которого вбит гвоздь, оставляющий на окружности отметины.\\
\textbf{a)} Докажите, что в какой-то момент новые отметины перестанут появляться.\\
 \textbf{b)} Докажите, что в этот момент отметины делят обод неподвижного колеса на равные части.\\ 
\textbf{c)} Пусть $c$~--- длина отрезка между двумя соседними отметинами. Докажите, что $c=1$.\\ 
\textbf{d)} \textbf{(Линейное представление НОД)} Докажите, что существуют целые числа $m$ и $n$ такие, что $m a+n b=1$.\\
\textbf{e)} Докажите, что для произвольных чисел $a$ и $b$ (не обязательно взаимнопростых) существуют целые числа $m$ и $n$, такие что $am+bn=(a,b)$.
\item В государстве имеют хождение монеты достоинством $a$ и $b$ золотых, где $a$ и $b$ -- взаимнопростые натуральные числа. Докажите, что такими монетами можно (возможно, со сдачей) набрать любую сумму.

\item Докажите, что угол в $17^{\circ}$ можно разделить с помощью циркуля и линейки на $17$ равных частей.

\item В классе химии имеются $25$ пробирок объема $1$, $2$, \ldots, $25$ мл. Когда химики стали собираться в ЛМШ, оказалось, что у них осталось мало места, и они могут взять только набор из $10$ пробирок. Химики хотят, чтобы с помощью любых двух пробирок из набора можно было отмерить $1$ мл. Сколькими способами можно составить такой набор?

\item Петя произвольным образом разложил некоторое количество монет по $30$ коробкам.
Вася может выбрать любые $k$ коробок и добавить в каждую из них по монете. При каких
$k$ Вася такими операциями сможет при любом исходном раскладе уравнять число монет
во всех коробках?

\item \textbf{a)} Числа $a$, $b$ и $c$ взаимнопросты в совокупности, то есть $(a,b,c)=1$.\\
Докажите, что любое число $d$ можно представить в виде $d=ax+by+cz$, где $x$, $y$, $z$~--- целые.\\
\textbf{b)} Сформулируйте и докажите аналогичную задачу, если изначально дано $n$ чисел, взаимнопростых в совокупности.

\item Докажите, что для любых натуральных чисел $a$ и $b$ существуют такие натуральные числа $c$ и $d$, что числа $an+c$ и $bn+d$ взаимнопросты при всех натуральных $n$.

%\item На доске написано четыре числа $m$, $n$, $m$, $n$. Каждую минуту с числами проделывается следующая операция: если на доске записаны $x$, $y$, $z$, $t$ и $x>y$, то все числа стираются и вместо них записываются $x-y$, $y$, $z+t$, $t$, а если $x<y$, то вместо них записываются $x$, $y-x$, $z$, $t+z$. Через несколько минут первые два числа стали равны. Докажите, что сумма двух последних равна $2mn$. 

\end{problems}
%\item Натуральные числа $a, b$ и  $c$ удовлетворяют равенству:
%\begin{equation*}
%[a,c]-[b,c]= a-b
%\end{equation*}
%Докажите, что $a$, $b$ делятся на $c$. 
%Докажите, что в вершинах любого графа можно расставить натуральные числа так, что любые два числа,
%соединенные ребром имеют НОД>1, а числа, не соединенные ребром, взаимно просты.

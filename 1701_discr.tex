\renewcommand{\baselinestretch}{0.8}
\parskip=0.75\parskip

\begin{center}
\textbf{\Large Дискретная непрерывность}\\
\textit{17.07.16}
\end{center}

\epigraph{\it --- Я ранее бывал в этом городе, тогда даже ни одной гостиницы не было. А теперь сразу две и с интересными названиями!}{С. Карамов, ``Бег на месте, или замкнутый круг''}

Разбор.
В ряд выложены 100 черных и 100 красных шаров, причём самый левый и самый правый шары чёрные. Докажите, что можно выбрать слева подряд несколько шаров (но не все!) так, чтобы среди них количество красных равнялось количеству чёрных. 

\begin{problems}
\item Шеренга новобранцев стояла лицом к сержанту. По команде ``налево'' некоторые повернулись налево, некоторые~--- направо, а остальные~--- кругом. Всегда ли сержант сможет встать в строй так, чтобы с обеих сторон от него оказалось поровну новобранцев, стоящих к нему лицом? 

\item За круглым столом сидит 10 мальчиков и 10 девочек. Докажите, что найдётся группа из 10 сидящих подряд детей, в которой девочек и мальчиков поровну. 

\item
Существуют ли сто последовательных натуральных чисел, среди которых
ровно пять простых?

%\item   Некто   расставил   в   произвольном   порядке   десятитомное   собрание   сочинений.
%Назовем <<беспорядком>> пару томов (не обязательно соседних), в которой том с большим
%номером стоит левее. Для некоторой расстановки томов подсчитано количество всех
%<<беспорядков>>. Какие значения оно может принимать? 

\item За круглым столом сидит чётное количество гномов. У каждого на колпаке по несколько помпонов. Причём у любых двух рядом сидящих гномов количество помпонов отличается не более чем на 1. Докажите, что найдётся пара гномов, сидящих друг напротив друга, количества помпонов на колпаках которых отличаются не больше, чем на 1.

\item      В зале находятся $n$ юношей и $n$ девушек, причём никакие трое не находятся на одной прямой. Всегда ли можно провести по полу прямую черту так, чтобы в каждой из образовавшихся частей зала юношей и девушек было поровну (но не ноль)? 

\item Грани восьми единичных кубиков окрашены в чёрный и белый цвета так, что чёрных и белых граней поровну. Докажите, что из этих кубиков можно сложить куб со стороной 2, на поверхности которого чёрных и белых квадратиков поровну.

\item В ряд стоят $n$ шаров, известно, что среди этих шаров могут быть шары только двух цветов: чёрного и белого. Если в автомат отправить запрос $(a, b)$, то он проверит шары на местах с номерами $a$ и $b$ и если первый чёрный, а второй~--- белый, то поменяет их местами. Паша утверждает, что может, не зная исходного набора, написать такой список запросов, что потом на двух, известных ему позициях, будут шары одного цвета. Докажите, что Паша хвастается безосновательно.  


\end{problems}

\renewcommand{\baselinestretch}{1}
\parskip=1.25\parskip

\begin{center}
\Large
\textbf{Разнобой-1. 5 июля.}
\end{center}

\begin{enumerate}
\large
\item Сережа написал на доске три натуральных числа, а затем вычислил их попарные НОДы и НОКи. Могла ли сумма шести полученных чисел оказаться равной 1001?

\item Сколько есть прямоугольников из клеток на шахматной доске?

\item В какое наименьшее количество цветов можно покрасить клетки таблицы $4\times4$ (каждая клетка может быть покрашена только в один цвет) так, чтобы для любых различных двух цветов нашлись две клетки, которые покрашены в эти цвета и имеют общую сторону?

\item В треугольнике $ABC$ с $\angle A = 120^{\circ}$ проведены биссектрисы $AA_1$, $BB_1$ и $CC_1$. Прямые $AC_1$ и $BB_1$ пересекаются в точке $T$. Найдите углы треугольника $A_1B_1T$.

\item Существует ли степень двойки, из которой перестановкой цифр (0 ставить на первое место нельзя) можно получить другую степень двойки?


\item Докажите, что уравнение $x^{3}+y^{3}=7\cdot 8^{k}$ не имеет решений в натуральных числах.

\item На плоскости даны $2n$ точек. Два игрока по очереди выбирают по
одной точке до тех пор, пока они не закончатся. Проигрывает тот, у
кого сумма попарных расстояний между выбранными им точками меньше,
чем у соперника. Кто выиграет при правильной игре -- начинающий или
его партнёр? (Все расстояния между данными точками и все суммы
попарных расстояний для разных наборов точек попарно различны.)

\end{enumerate}


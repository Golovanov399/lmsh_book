\begin{center}
\Large
\textbf{Индукция. 9 июля. }
\end{center}  

\bigskip
\large
{\bf Можно сдавать только решения с помощью метода математической  индукции, даже если очень хочется по-другому.}

\bigskip

\q1. Докажите, что $1^2+3^2+\ldots +(2k-1)^2 = \frac{k(4k^2-1)}{3}$.

\quad

\q2. Определим число $n?$ ("эн вопросиал") следующим образом: $1? = 1$, 
$n?={n\over (n-1)?}$ для всех $n>1$. Докажите, что $n?\times n!$ -- 
квадрат натурального числа.

\quad


\q3. Для каких натуральных $n$ выполнено неравенство $2^n>n^3$?


\quad




\q4. Обозначим за $P(n)$ произведение всех простых чисел, меньших $n$. Докажите, что если $n>3$, то $P(n)>n$.

 
\quad




\q5. Вася написал на $n!$ бумажках все возможные последовательности, содержащие числа от 1 до $n$ по одному разу. Докажите, что можно выложить бумажки по кругу так, чтобы последовательности, написанные на соседних бумажках, отличались лишь перестановкой двух соседних чисел. 


\quad


\q6. В военную часть приехало $n$ незнакомых друг с другом новобранцев. Прапорщик сообщил каждому новобранцу натуральное число так, что сумма всех $n$ чисел равна $2n-2$. Докажите, что можно познакомить некоторых из них друг с другом так, чтобы каждый 
новобранец имел количество знакомых, равное числу, которое ему сообщил прапорщик.

\quad

\q7. На ежегодный слет-проверку съехались 100 бригад СЭС и поселились в 100 корпусах. Каждая бригада хочет проверить три корпуса (заранее сообщая номера начальнику лагеря) и тут же уехать из лагеря. Докажите, что начальник может составить расписание проверок так,  чтобы никакую бригаду не проверяли более трех раз (проверять пустые корпуса можно сколько угодно раз).

\quad




\q8. В стране $n$ городов. Каждые два города соединены дорогой. По указу Парламента Министерству транспорта необходимо вести учёт всех циклических туристических маршрутов. Докажите, что министерство транспорта может так ввести одностороннее движение на дорогах, чтобы из любого города можно было доехать до любого другого и для каждого $k\leqslant n$ циклических маршрутов длины $k$ было ровно $n-k+2$ (то есть, чтобы учёта нужно было вести как можно меньше). 

\quad

\q9. В каждой клетке таблицы $1000\times 1000$ стоит ноль или единица. Докажите, что можно либо вычеркнуть 990 строк так, что каждом столбце будет хотя бы одна невычеркнутая единица, либо вычеркнуть 990 столбцов так, что в каждой строке будет хотя бы один невычеркнутый ноль.
